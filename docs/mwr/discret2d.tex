
%------------------------------------------------------------------------
%
%    Copyright (C) 1985-2020  Georg Umgiesser
%
%    This file is part of SHYFEM.
%
%    SHYFEM is free software: you can redistribute it and/or modify
%    it under the terms of the GNU General Public License as published by
%    the Free Software Foundation, either version 3 of the License, or
%    (at your option) any later version.
%
%    SHYFEM is distributed in the hope that it will be useful,
%    but WITHOUT ANY WARRANTY; without even the implied warranty of
%    MERCHANTABILITY or FITNESS FOR A PARTICULAR PURPOSE. See the
%    GNU General Public License for more details.
%
%    You should have received a copy of the GNU General Public License
%    along with SHYFEM. Please see the file COPYING in the main directory.
%    If not, see <http://www.gnu.org/licenses/>.
%
%    Contributions to this file can be found below in the revision log.
%
%------------------------------------------------------------------------

% changed for 2d semi-implicit explanations

%%%%%%%%%%%%%%%%%%%%%%%%%%%%%%%%%%%%%%%%%%%%%%%%%%%%%%%%%%
%%%%%%% basic heading for Latex %%%%%%%%%%%%%%%%%%%%%%%%%%
%%%%%%%%%%%%%%%%%%%%%%%%%%%%%%%%%%%%%%%%%%%%%%%%%%%%%%%%%%

\documentclass[12pt]{article}

\usepackage{a4}
%\usepackage{amssymb}
%\usepackage{amsmath}




%%%%%%%%%%%%%%%%%%%%%%%%%%%%%%%%%%%%%%%%%%%%%%%%%%%%%%%%%%
%%%%%%% user commands %%%%%%%%%%%%%%%%%%%%%%%%%%%%%%%%%%%%
%%%%%%%%%%%%%%%%%%%%%%%%%%%%%%%%%%%%%%%%%%%%%%%%%%%%%%%%%%

\newcommand{\vect}[2]   {       \mbox{$
					\left(  \begin{array}{c}
					#1 \\ #2
					\end{array}     \right)
				$}
			}
\newcommand{\paren}[1]  { \left( #1 \right) }

\newcommand{\nsz} {\normalsize}

%<MATH_ABBREV>

\newcommand{\tdif}[1] {\frac{\partial #1}{\partial t}}
\newcommand{\xdif}[1] {\frac{\partial #1}{\partial x}}
\newcommand{\ydif}[1] {\frac{\partial #1}{\partial y}}
\newcommand{\zdif}[1] {\frac{\partial #1}{\partial z}}
\newcommand{\xxdif}[1] {\frac{\partial^2 #1}{\partial x^2}}
\newcommand{\yydif}[1] {\frac{\partial^2 #1}{\partial y^2}}
\newcommand{\zzdif}[1] {\frac{\partial^2 #1}{\partial z^2}}

\newcommand{\dt} {\Delta t}
\newcommand{\dtt} {\mbox{$\frac{\Delta t}{2}$}}
\newcommand{\dx} {\mbox{$\Delta x$}}
\newcommand{\dy} {\mbox{$\Delta y$}}
\newcommand{\dz}{\Delta z}


\newcommand{\olds} {\mbox{$\scriptstyle (0)$}}
\newcommand{\news} {\mbox{$\scriptstyle (1)$}}
\newcommand{\meds} {\mbox{$\scriptscriptstyle (\frac{1}{2})$}}
\newcommand{\half} {\mbox{$\scriptstyle \frac{1}{2}$}}

\newcommand{\uold} {\mbox{$U^{\olds}$}}
\newcommand{\vold} {\mbox{$V^{\olds}$}}
%\newcommand{\zold} {\mbox{$\zeta^{\olds}$}}
\newcommand{\unew} {\mbox{$U^{\news}$}}
\newcommand{\vnew} {\mbox{$V^{\news}$}}
%\newcommand{\znew} {\mbox{$\zeta^{\news}$}}

\newcommand{\uuold} {\mbox{$u^{\olds}$}}
\newcommand{\vvold} {\mbox{$v^{\olds}$}}
\newcommand{\uunew} {\mbox{$u^{\news}$}}
\newcommand{\vvnew} {\mbox{$v^{\news}$}}

\newcommand{\uzl} {\mbox{$\zdif{U_{l}}$}}
\newcommand{\vzl} {\mbox{$\zdif{V_{l}}$}}

\newcommand{\uoldp}[1] {\mbox{$U_{{#1}}^{\olds}$}}
\newcommand{\voldp}[1] {\mbox{$V_{{#1}}^{\olds}$}}
\newcommand{\zoldp}[1] {\mbox{$\zeta_{{#1}}^{\olds}$}}
\newcommand{\unewp}[1] {\mbox{$U_{{#1}}^{\news}$}}
\newcommand{\vnewp}[1] {\mbox{$V_{{#1}}^{\news}$}}
\newcommand{\znewp}[1] {\mbox{$\zeta_{{#1}}^{\news}$}}
\newcommand{\zmedp}[1] {\mbox{$\zeta_{{#1}}^{\meds}$}}

\newcommand{\resr} {{\cal R}}
\newcommand{\rhon} {\rho_0}
\newcommand{\drho} {\frac{1}{\rhon}}
\newcommand{\fracs}[2] {\mbox{$\frac{#1}{#2}$}}
%\newcommand{\drho} {\mbox{$\scriptstyle \frac{1}{\rho_{0}}$}}
\newcommand{\deltat} {\mbox{$\tilde{\delta}$}}
\newcommand{\gammat} {\mbox{$\tilde{\gamma}$}}

\newcommand{\AO} {A_{\Omega}}
\newcommand{\dO} {d \Omega}

\newcommand{\uv} {{\bf U}}
\newcommand{\uvold} {{\bf U^{(0)}}}
\newcommand{\uvnew} {{\bf U^{(1)}}}

\newcommand{\Uold} {U^{(0)}}
\newcommand{\Unew} {U^{(1)}}
\newcommand{\Vold} {V^{(0)}}
\newcommand{\Vnew} {V^{(1)}}
\newcommand{\zold} {\zeta^{(0)}}
\newcommand{\znew} {\zeta^{(1)}}

\newcommand{\Xb} {\hat{X}}
\newcommand{\Yb} {\hat{Y}}
\newcommand{\aush} {A_H}

\newcommand{\az} {\alpha_{z}}
\newcommand{\am} {\alpha_{m}}
\newcommand{\ar} {\alpha_{r}}
\newcommand{\azt} {\tilde{\az}}
\newcommand{\amt} {\tilde{\am}}
\newcommand{\art} {\tilde{\ar}}

\newcommand{\duv} {\Delta {\bf U}}
\newcommand{\dzeta} {\Delta \zeta}
\newcommand{\iv} {{\bf I}}
\newcommand{\ivh} {\hat{\bf I}}
\newcommand{\fv} {{\bf F}}
\newcommand{\uvh} {\hat{\bf U}}
\newcommand{\ffxx} {\tilde{f_x}}
\newcommand{\ffyy} {\tilde{f_y}}

\newcommand{\dvol} {\Delta v}
\newcommand{\cbar} {\bar c}

\newcommand{\tv} {\tau}

%</MATH_ABBREV>

\newcommand{\beq} {\begin{equation}}
\newcommand{\eeq} {\end{equation}}
\newcommand{\beqa} {\begin{eqnarray}}
\newcommand{\eeqa} {\end{eqnarray}}

\newcommand{\eqref} [1] {(\ref{eq:#1})}

%%%%%%%%%%%%%%%%%%%%%%%%%%%%%%%%%%%%%%%%%%%%%%%%%%%%%%%%%%
%%%%%%% hyphenation %%%%%%%%%%%%%%%%%%%%%%%%%%%%%%%%%%%%%%
%%%%%%%%%%%%%%%%%%%%%%%%%%%%%%%%%%%%%%%%%%%%%%%%%%%%%%%%%%

%%%%%%%%%%%%%%%%%%%%%%%%%%%%%%%%%%%%%%%%%%%%%%%%%%%%%%%%%%
%%%%%%% title & author %%%%%%%%%%%%%%%%%%%%%%%%%%%%%%%%%%%
%%%%%%%%%%%%%%%%%%%%%%%%%%%%%%%%%%%%%%%%%%%%%%%%%%%%%%%%%%

\title{Discretization of 2D FEM model}

\author{Georg Umgiesser}

%\date{\today}
%\date{}

%%%%%%%%%%%%%%%%%%%%%%%%%%%%%%%%%%%%%%%%%%%%%%%%%%%%%%%%%%
%%%%%%% document %%%%%%%%%%%%%%%%%%%%%%%%%%%%%%%%%%%%%%%%%
%%%%%%%%%%%%%%%%%%%%%%%%%%%%%%%%%%%%%%%%%%%%%%%%%%%%%%%%%%

\begin{document}

%\baselineskip 20 pt
\maketitle

%\begin{abstract}
%The main features of a newly conceived primitive equation circulation
%model are outlined. The model uses finite elements for spatial
%integration and a semi-implicit algorithm for integration in time.
%The terms treated implicitly are the water levels, the Coriolis term
%and the vertical eddy diffusion. The model uses a combination of
%linear form functions for the expansion of the
%water levels and constant form functions for the velocities.
%Through this combination of form functions the resulting grid resembles
%a staggered finite difference Arakawa B-grid. This
%staggered finite element grid is superior, as far as conservation
%and propagation properties are concerned, to a standard Galerkin method.
%The model has already been used successfully with
%the Venice Lagoon and
%in its full three dimensional formulation will be also
%applied to the Adriatic Sea.
%\end{abstract}


\section*{The Equations}


We start with the two-dimensional
shallow water momentum equations and the continuity equation:
\beq
\tdif{u} + g \xdif{\zeta} + \frac{1}{\rho}\zdif{\tau_x} + \tilde{X} = 0
\eeq
\beq
\tdif{v} + g \ydif{\zeta} + \frac{1}{\rho}\zdif{\tau_y} + \tilde{Y} = 0
\eeq
\beq
\xdif{u} + \ydif{v} + \zdif{w} = 0
\eeq
The terms $\tau_x$ and $\tau_y$ are the stress terms.

Integrating over the water column gives
\beq
\tdif{U} + gH \xdif{\zeta} + RU + X = 0
\eeq
\beq
\tdif{V} + gH \ydif{\zeta} + RV + Y = 0
\eeq
\beq
\tdif{\zeta} + \xdif{U} + \ydif{V} = 0
\eeq
where $x,y,t$ are the space coordinates and time, $U,V$ the
transports in $x,y$ direction, $\zeta$ the water level,
$R$ the friction parameter, 
$g$ the gravitational
acceleration, $H=h+\zeta$ the total water depth, $h$ the
undisturbed water depth and
$X,Y$ extra terms that can be treated
explicitly in the following discretization and as explained later.

wwwww

wwwww

\section*{The Friction}

The friction parameter $R$ depends itself on the
current velocity. If a quadratic bottom friction is used we have
\beq
\tau_x^B = c_B \rho |u| u
\eeq
\beq
\tau_y^B = c_B \rho |u| v
\eeq
and $R = \frac{c_B |u|}{H}$.

The transports $U,V$ can be obtained from the velocities by
\beq
	U = \int_{-h}^{\zeta} u dz 
		\hspace{1cm} 
	V = \int_{-h}^{\zeta} v dz
\eeq
where $u,v$ are the current velocities.

The terms contained in $X,Y$ are the non-linear advective terms, the
Coriolis terms, the wind stress and the lateral turbulent diffusion.
They may be written as
\beq
X = U \xdif{u} + V \ydif{u} - f V - \frac{\tau^x}{\rhon}
	- \aush ( \xxdif{U} + \yydif{U} )
\eeq
\beq
Y = U \xdif{v} + V \ydif{v} + f U - \frac{\tau^y}{\rhon}
	- \aush ( \xxdif{V} + \yydif{V} )
\eeq
where $f$ is the Coriolis parameter, $\tau^x,\tau^y$ the
wind stress, $\rhon$ the reference density of water and
$\aush$ the horizontal eddy viscosity.


\section*{Discretization of the Momentum Equation}


The various time varying variables are now discretized in time.
The general expression for this is 
$\zeta = \alpha \znew + \tilde{\alpha} \zold$ where the index
$\news$ designs the new and $\olds$ the old time level.
The weighting parameters $\alpha,\tilde{\alpha}$ must be in the
the intervall from 0 to 1 and $\alpha+\tilde{\alpha}=1$ must hold.

We now chose weighting parameters for the discretization. These parameters
are $a_z$ for the transports in the continuity equation, $a_m$
for the pressure term in the momentum equations and $a_r$ for
the friction term. Associated to these parameters are the parameters
$\azt, \amt, \art$ that are defined as
\beq
\azt = 1 - a_z \quad \amt = 1 - a_m \quad \art = 1 - a_r.
\eeq
All above parameters can take the values from 0 to 1 where
0 means an explicit treatment and 1 a complete implicit treatment.

Discretizing the $x$ momentum equation one obtains
\beq
\frac{\Unew-\Uold}{\dt} + gH [ \am \xdif{\znew} + \amt \xdif{\zold} ]
	+ R [ \ar \Unew + \art \Uold ] + X = 0
\eeq
where the total depth $H$ and the extra terms $X$ are always taken
at the old time step.

Solving for $\Unew$ and introducing the new parameter 
\beq
\delta = \frac{1}{1 + \dt R \ar}
\eeq
we obtain 
\beq
\Unew = \delta (1 - \dt R \art) \Uold 
		- \dt \delta gH [ \am \xdif{\znew} + \amt \xdif{\zold} ]
		- \dt \delta X.
\eeq

Introducing two more auxiliary parameters
\beq
	\gamma = \delta [ 1 - \dt R \art ] \quad
	\beta = \dt \delta gH
\eeq
we finally have for both momentum equations
\beq
\Unew = \gamma \Uold 
		- \beta \am \xdif{\znew} - \beta \amt \xdif{\zold} 
		- \dt \delta X
\eeq
\beq
\Vnew = \gamma \Vold 
		- \beta \am \ydif{\znew} - \beta \amt \ydif{\zold} 
		- \dt \delta Y
\eeq
where the equation in $y$ direction
has been obtained in a similar way 
as the one in $x$ direction.

\section*{Discretization of the continuity equation}

\beq
\tdif{\zeta} + \xdif{U} + \ydif{V} = 0
\eeq
\beq
\znew = \zold 
		- \dt [ \am \xdif{\Unew} + \amt \xdif{\Uold} ]
		- \dt [ \am \ydif{\Vnew} + \amt \ydif{\Vold} ]
\eeq

We can insert formally $\Unew$ and $\Vnew$ into this equation and obtain
an equation in $\znew$ that can be solved by solving a linear system.

\section*{The tre-dimensional case}

We can again, 

\begin{equation}
	\tdif{U_l} + gH_l\xdif{\zeta}
		= \tilde{F}_{l} + \nu\zzdif{U_l}
\end{equation}
\begin{equation}
	\tdif{V_l} + gH_l\ydif{\zeta}
		= \tilde{F}_{l} + \nu\zzdif{V_l}
\end{equation}
and the continuity equation
\begin{equation}
	\tdif{\zeta}
		+ \sum_l \xdif{U_l}
		+ \sum_l \ydif{V_l}
		= 0
\end{equation}

We obtain a similar equation as before, but now we have the unknown velocity
$U_L$ and $V_L$ which are the transports in the bottom layer. Now
we cannot solve for $U,V$ anymore as before.

\beq
\tdif{U} + gH \xdif{\zeta} + RU_L + X = 0
\eeq
\beq
\tdif{V} + gH \ydif{\zeta} + RV_L + Y = 0
\eeq
\beq
\tdif{\zeta} + \xdif{U} + \ydif{V} = 0
\eeq

\beq
        \frac{\Unew_l-\Uold_l}{\dt}
        - f(\alpha_f\Vnew_l + (1-\alpha_f)\Vold_l)
        + gH_l \xdif(\alpha_m\znew + (1-\alpha_m)\zold)
        = \\
        F^x_{l}
        + \nu\alpha_d\frac{\Unew_{l-1}-2\Unew_l+\Unew_{l+1}}{\dz^2}
        + \nu(1-\alpha_d)\frac{\Uold_{l-1}-2\Uold_l+\Uold_{l+1}}{\dz^2}
\eeq

\beq
        \frac{\Vnew_l-\Vold_l}{\dt}
        + f(\alpha_f\Unew_l + (1-\alpha_f)\Uold_l)
        + gH_l \ydif(\alpha_m\znew + (1-\alpha_m)\zold)
        = \\
        F^y_{l}
        + \nu\alpha_d\frac{\Vnew_{l-1}-2\Vnew_l+\Vnew_{l+1}}{\dz^2}
        + \nu(1-\alpha_d)\frac{\Vold_{l-1}-2\Vold_l+\Vold_{l+1}}{\dz^2}
\eeq

\section*{Spatial Integration of the Momentum Equation}


We now carry out the integration of the momentum equation over
one finite element. For this the variable are expanded with their
respective form functions, e.g., $\zold = \zold_M \Phi_M$ where
$M=1,3$ for a triangular element and equal indices
imply a summation over these indices.

We multiply every term with the constant weighting function
$\Psi$ and integrate. Remember that $U,V$ are constant over an
element, and $\zeta$ is varying linearly. The single terms
give
\beq
\int \Psi \Unew \dO = \AO \Unew
\eeq
\beq
\int \Psi X \dO = \int X \dO
\eeq
\beq
	\int \Psi \xdif{\znew} \dO 
	= \int \Psi \xdif{\Phi_M} \znew_M \dO 
	= \int b_M \znew_M \dO = \AO b_M \znew_M
\eeq
\beq
	\int \Psi \ydif{\znew} \dO 
	= \int \Psi \ydif{\Phi_M} \znew_M \dO 
	= \int c_M \znew_M \dO = \AO c_M \znew_M
\eeq
where $\Omega$ is the integration domain, $\AO$ the area of the
triangle and $b_M, c_M$ are the constant derivatives of the 
linear form functions $\Phi_M$
\beq
b_M = \xdif{\Phi_M} \quad c_M = \ydif{\Phi_M}.
\eeq

We therefore obtain
\beq
\Unew = \gamma U 
		- \beta \am b_M \znew_M - \beta \amt b_M \zold_M 
		- \dt \delta \Xb
\eeq
\beq
\Vnew = \gamma V 
		- \beta \am c_M \znew_M - \beta \amt c_M \zold_M 
		- \dt \delta \Yb
\eeq
with
\beq
\Xb = \frac{1}{\AO} \int X \dO \quad
\Yb = \frac{1}{\AO} \int Y \dO.
\eeq


\section*{Integration of the Continuity Equation}





Before discretizing we first multiply the continuity
equation by the form function $\phi_N$ and integrate
over one element to obtain
\[
\int \Phi_N \tdif{\zeta} + \int \Phi_N \xdif{U} + \int \Phi_N \ydif{V} = 0.
\]
The terms containing the spatial differentiation can now be transformed
by partial integration in the following way
\[ 
        \int \Phi_N \xdif{U} + \int \Phi_N \ydif{V} =
        - \int \xdif{\Phi_N} U - \int \ydif{\Phi_N} V +
                \int_{\partial\Omega} \Phi_N U_n ds.
\]
and remembering that the partial derivatives of the form functions
$\Phi_N$ are $b_N,c_N$ respectively we can write
\beq \label{eq:conti1}
        \int \Phi_N \xdif{U} + \int \Phi_N \ydif{V} =
        - \int b_N U - \int c_N V +
                \int_{\partial\Omega} \Phi_N U_n ds.
\eeq

Here the last integral is a line integral over the border of
the element and $U_n$ the transport normal to the border. Once the
various contributions of the elements are summed the terms of the last
integral are equal in modulus but of inverse sign and therefore
cancel out. The only terms that do not cancel out are the
line integral over the border of the domain.

In case of a material boundary the trasnport noram lto the boundary
vanishes, and therefore the line integral over these elements
is again zero. The only contribution different from zero is therefore
at the open boundaries. In this case the line integral is exactly
the discharge over the open boundary out of the domain. It is
therefore natural to specify a flux boundary condition with finite elements.
However, in case the water levels are specified this boundary
condition is not needed and is substituted with a level
boundary condition. We will not further consider this
term from now on.


The integrated continuity equation \eqref{conti1} is now discretized in time
in the following way
\[
        \int \frac{\znew-\znew}{\dt} \dO =
                b_N \int ( \az \unew + \azt \uold ) \dO +  
                c_N \int ( \az \vnew + \azt \vold ) \dO
\]






\end{document}
