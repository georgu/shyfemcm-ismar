
%------------------------------------------------------------------------
%
%    Copyright (C) 1985-2020  Georg Umgiesser
%
%    This file is part of SHYFEM.
%
%    SHYFEM is free software: you can redistribute it and/or modify
%    it under the terms of the GNU General Public License as published by
%    the Free Software Foundation, either version 3 of the License, or
%    (at your option) any later version.
%
%    SHYFEM is distributed in the hope that it will be useful,
%    but WITHOUT ANY WARRANTY; without even the implied warranty of
%    MERCHANTABILITY or FITNESS FOR A PARTICULAR PURPOSE. See the
%    GNU General Public License for more details.
%
%    You should have received a copy of the GNU General Public License
%    along with SHYFEM. Please see the file COPYING in the main directory.
%    If not, see <http://www.gnu.org/licenses/>.
%
%    Contributions to this file can be found below in the revision log.
%
%------------------------------------------------------------------------

Here is a list of packages, tools and libraries that you will need to build and
run the \shy{} model.
Here below we use the command |sudo apt install| to install a package in a Debian
based linux system (installation is tested on Ubuntu) but other Linux
distributions can be considered.

\subsection{Mandatory}
\begin{itemize}

\item The package |make| is required for compilation.

\item The |perl| interpreter and the |bash| shell are necessary for compiling.

\item A Fortran 77 and 90 compiler is required. Supported compilers are the Gnu
compiler |gfortran| and the Intel Fortran compiler |ifort|.

\item A C compiler. Supported compilers are the Gnu |gcc| and the Intel C
compiler |icc|.

\end{itemize}

\subsection{Optional}

\begin{itemize}

\item For parallel runs: the mesh partitioner |METIS| and the |MPI| library are
 required.

\item For output files in NetCDF, the |NetCDF| library is necessary.

\item For solving the linear system in parallel, the |PETSC| solver is necessary
%(to install it refer to \ref{PETSC}).

\end{itemize}

%\textcolor{red}{Continue with others (cf output check_software)? graphics...?} \\
Note that once \shy{} has been downloaded (see section \ref{inst_where}), it is possible to
automatically check the status of the programs listed here above (plus some more additional packages)
through the command:
\begin{code}
   make check_software
\end{code}
If you get something like |bash: make: command not found|, then you do
not have make installed. Please first install the |make| command and
then run the command again.

The output of the command |make check_software| will show you which software you still have
to install.

If you need additional packages, the name of the corresponding package
to install can be found at the
web-page\footnote{https://www.debian.org/distrib/packages} for Debian OS.
Usually, Debian-based (e.g., Ubuntu) distributions have the same name.

Whereas most package names are easy to guess, a common problem refers to
 the developer X11 libraries. In order to be able to compile the
program |grid| you will need to install some packages that may have
different names depending on your distribution. The packages you need to find
 are |libx11-dev|, |x11proto-core-dev| and |libxt-dev|.

Please note that you have to carry out the steps in this section only
the first time you install the model. If you install a new version of
\shy{} software you can skip these steps.

