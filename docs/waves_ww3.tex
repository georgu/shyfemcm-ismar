
%------------------------------------------------------------------------
%
%    Copyright (C) 1985-2020  Georg Umgiesser
%
%    This file is part of SHYFEM.
%
%    SHYFEM is free software: you can redistribute it and/or modify
%    it under the terms of the GNU General Public License as published by
%    the Free Software Foundation, either version 3 of the License, or
%    (at your option) any later version.
%
%    SHYFEM is distributed in the hope that it will be useful,
%    but WITHOUT ANY WARRANTY; without even the implied warranty of
%    MERCHANTABILITY or FITNESS FOR A PARTICULAR PURPOSE. See the
%    GNU General Public License for more details.
%
%    You should have received a copy of the GNU General Public License
%    along with SHYFEM. Please see the file COPYING in the main directory.
%    If not, see <http://www.gnu.org/licenses/>.
%
%    Contributions to this file can be found below in the revision log.
%
%------------------------------------------------------------------------

SHYFEM has been coupled to the WAVEWATCH III (WW3) wave model.
The SHYFEM model was coupled to WW3 based on a so called ``hard coupling'', e.g.
binding the models directly without any so called coupling library such as
(ESMF, OASIS, PGMCL or others). The benefit is on the hand, minimal memory
usage, very limited code changes in both models. Top-level approach using the
coupled model framework based on having a routine for initialization,
computation and finalization, that allows a neat inclusion of WW3 in any kind
of flow model.

\paragraph{Installation and compilation}
%The coupled model is disributed within the SHYFEM distribution and the WW3 code
%is located in the |shyfem/WW3| subfolder. The latest version of the WW3 code is
%available on GitHub at \url{https://github.com/NOAA-EMC/WW3}.

In order to compile the coupled SHYFEM-WW3 model, the variable |WW3| in the 
file |Rules.make| should be set as |WW3 = true| and |WW3DIR| should
point to the folder containing the |WW3| code (e.g., WW3DIR = \$(HOME)/bin/WW3). 
|SHYFEM| should be compiled for running in parallel by setting 
|PARALLEL_MPI = NODE|.

Moreover, WW3 also needs additional libraries:
\begin{itemize}
\item |netcdf| compiled with the same compiler.  You must set |NETCDF = true|
and specify the variable |NETCDFDIR| to indicate the directory where the 
libraries and its include files can be found.
\item |METIS| and |PARMETIS| compiled with the same compiler. You must set
|PARTS = PARMETIS| and set the variables |METISDIR| and |PARMETISDIR|
to indicate the directory where the libraries and its include files can be
found.
\end{itemize}

The compilation of |WW3| can be customized changing a set of model options,
defined with the variable |WW3CFLAGS| in the file |fem3d/Makefile|. They
correspond to the parameters listed in the file |switch| needed by the stand 
alone WW3 installation. For a detailed description of the options see the WW3 
manual (section 5.9).

You can now follow the general |SHYFEM| installation instruction to compile the 
code.

It is however required to download, compile and install the stand alone |WW3| 
model from the ERDC GitHub at |https://github.com/erdc/WW3/tree/ww3_shyfem|. 
This step is needed for having |WW3| source code and pre- and post-processing 
tools (e.g., ww3\_grid). 
It is important that the so called ``switch'' file of |WW3| (see |WW3| documentation) 
contains the same parameters listed with variable |WW3CFLAGS| are identical 
for both modules. The default setup provided with |WW3CFLAGS| is basically 
identical to the setup of SHOM and Meteo France as well as the USACE based 
on implicit time stepping on unstructured grids. Basically, there is no need 
for modification of the ``switch'' file with the given settings but all 
settings are supported as long the unstructured grid option in WW3 is used.

For installing and comping the stand alone |WW3| model you can follow
the following procedure (see the |WW3| manual for more informations):
\begin{enumerate}
\item download the ww3\_shyfem branch from the ERDC git repository: 
git clone --branch ww3\_shyfem https://github.com/erdc/WW3/tree/ww3\_shyfem;
\item move to the WW3 directory;
\item set the NetCDF path: with |export WWATCH3_NETCDF=NC4| and \\
|export NETCDF_CONFIG=/netcdf-dir/nc-config|;
\item setup the model using |./model/bin/w3_setup /home/model/bin/WW3/model -c| \\
| <cmplr> -s <swtch>| with |cmplr| the compiler |comp| and |link| files (e.g., shyfem
for files comp.shyfem and link.shyfem) and |swtch| the switch file (e.g., shyfem 
for a switch file located in ./model/bin having name switch\_shyfem).
\item compile |WW3| with |w3_automake|, or for a few programs with
|./model/bin/w3_automake| \\ |ww3_grid ww3_shel ww3_ounf| with name 
\item the compiled programms are located in |./model/exe/ww3_grid|
\end{enumerate}

\paragraph{Coupling description}
The implementation was done by developing two modules, one in
|SHYFEM| (subww3.f) and the other one in |WW3| (w3cplshymfen.F90). The 
SHYFEM coupling module is connected by the ``use'' statement to the 
modules of the |WW3| code. In this way one can access all fields from 
the |WW3| in |SHYFEM| and vice versa. In the |WW3| module all spectral based
quantities are computed and |SHYFEM| is directly assigning the various 
arrays, based on global arrays, to the certain domains for each of the 
model decompositions.

In terms of |WW3| we have used the highest-level implementation based
on the |WW3| multigrid driver. This allows basically to couple any kind
of grid type with |SHYFEM| (rectangular, curvi-linear, SMC and
unstructured). It offers of course the possibility to run |WW3| based
on a multigrid SETUP and coupled to |SHYFEM|. Both models use their
native input files for running the model except that the forcing by
wind within the coupled model is done via |SHYFEM| and the flow field
is by definition provided based on the coupling to |SHYFEM|. In this
way both models are fully compatible to the available documentation.

The two numerical models (SHYFEM and WW3) should exchange all the
variables that are needed to simulate the current-wave interaction.
The following variables sare passed by |SHYFEM| to |WW3|:
\begin{itemize}
\item water levels;
\item three-dimensional water currents;
\item wind components.
\end{itemize}

The following physical quantities have been computed in |WW3| and are 
available for |SHYFEM|:
\begin{itemize}
\item significant wave height;
\item mean wave period;
\item peak wave period;
\item main wave direction (the where the wave go);
\item radiation stress,
\item wave pressure;
\item wind drag coefficient;
\item Stokes velocities.
\end{itemize}

The exchange module of SHYFEM contains all the needed infrastructure
for initializing, calling and finalizing the wave model run. Two
subroutines (getvarSHYFEM and getvarWW3), which are called before and
after the call to the wave model in order to obtain currents, water
levels and on the other hand integral wave parameters and radiation
stresses.

The significant wave height, mean wave period and main wave 
direction are written by |SHYFEM| in the |.wave.shy| file according
to the values of the variables |idtwav| and |itmwav| in the |wave|
section of the |SHYFEM| parameter file (see below).

\paragraph{Running the coupled SHYFEM-WW3 model}
Since the wave output are written by |SHYFEM|, the output are
set to off in |WW3|, which reduces disk usage and significantly reduces 
the parallel overhead due to the output part. 

As we have utilized here the implicit scheme in WW3, which was
developed by Roland \& Partner and the implicit scheme is well
validated and unconditionally stable. The choice of the time step
should be according to the physical time scales of the modelled
processes. 

In order to run the coupled model, we suggest the following
procedure:
\begin{enumerate}
\item Setup the |SHYFEM| set-up for the region of interest. 
|SHYFEM| needs a mesh in the .bas format, a parameter file and 
all forcing files (see |SHYFEM| documentation). In the |SHYFEM| 
parameter file the section |\$waves| must be activated with |iwave = 11|.
The time step for coupling with WW3 |dtwave| must be set equal
to the |SHYFEM| and |WW3| timesteps (TO BE FIXED).
|idtwav|, |itmwav| should also be set for determining the 
time step and start time for writing to the output file |.wave.shy| .
Preprocess the |SHYFEM| grid with a selected number of domains 
without bandwidth optimization with |shypre -noopti grid.grd|.

\item Setup the |WW3| model for the region of interest. In the
running directoty, |WW3| requires:
\begin{enumerate}
\item the file |ww3_grid.nml| containing the spectrum, run, timesteps
and grid parameterizations (see |WW3| documentation). This file also
contains the names of mesh and namelist files.
\item the namelist file containing the variables for setting
the tunable parameters for source terms, propagation schemes, and 
numerics.
\item the file |ww3_multi.nml| which handles the time steps and 
the I/O. Since |WW3| receives flow and wind from |SHYFEM| there are 
no input fields that need to be prescribed. The output part of the 
wave model itself is handled by |SHYFEM| as well.
\item the numerical mesh in the GMSH format .msh. The |SHYFEM|
mesh in .grd format can be easily converted to .msh with the command:
shybas -msh grid.bas (it creates a bas.msh file).
\end{enumerate}

Before running the model, the setup |ww3_grid| tool (found in the
stand alone |WW3/build/bin| directory) needs to be run and the output, 
which is named per default ``mod\_def.ww3'' needs to be renamed 
with the extension defined in parameter |MODEL(1)\%NAME = 'med'|
in |ww3_multi.nml| (e.g., mod\_def.med in the above mentioned case). 
The |ww3_grid| tool must be run every time the namelist or grid files 
are modified. 

\item run the coupled model with the |shyfem| command. Since the code
is optimized to run with MPI on domain decomposition, we suggest to
run the coupled model in parallel using |mpiexec -np np shyfem namelist.str| 
with np the number of processors.

\end{enumerate}

