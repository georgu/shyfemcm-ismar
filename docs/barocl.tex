
%------------------------------------------------------------------------
%
%    Copyright (C) 1985-2020  Georg Umgiesser
%
%    This file is part of SHYFEM.
%
%    SHYFEM is free software: you can redistribute it and/or modify
%    it under the terms of the GNU General Public License as published by
%    the Free Software Foundation, either version 3 of the License, or
%    (at your option) any later version.
%
%    SHYFEM is distributed in the hope that it will be useful,
%    but WITHOUT ANY WARRANTY; without even the implied warranty of
%    MERCHANTABILITY or FITNESS FOR A PARTICULAR PURPOSE. See the
%    GNU General Public License for more details.
%
%    You should have received a copy of the GNU General Public License
%    along with SHYFEM. Please see the file COPYING in the main directory.
%    If not, see <http://www.gnu.org/licenses/>.
%
%    Contributions to this file can be found below in the revision log.
%
%------------------------------------------------------------------------

The baroclinic term permits to compute
the variation of the water density due to the horizontal
gradients of temperature and/or salinity. These variations
causes an additional motion of the water. 

In order to use the baroclinic term, the parameter |ibarcl| 
(in section |$para|) must be set different from 0.

Setting |ibarcl| to a value different from 0 will simulate the transport
and diffusion of temperature and salinity in the basin. A value of 1 will
compute the full baroclinic pressure terms, due to density gradients,
and the advection and diffution of temperature and salinity. A value of
2 will do diagnostic simulations. This means that baroclinic pressure
terms are still included in the hydrodynamic equations, but temperature
and salinity values will be read from a file.  Finally for |ibarcl|=3
temperature and salinity will be computed but no baroclinic pressure term
will be used. In this case the hydrodynamic equations and the equations
for temperature and salinity are decoupled and there is no feed back
from the water density field to the currents.

It is advisable to use a 3D computation with the non-linear terms and
a variable time step.  In any case, if temperature and salinity are
computed, first they must be initialized either with constant values
or with variable 3D matrices.  In the first case the reference values
have to be imposed in |temref| and |salref|. An example of this type of
simulation is given in \Fig\figref{baroc}.

If the temperature and salinity are given as 3D matrices files,
they must be provided in the |$name| section, giving the file 
names in |tempin| and |saltin|. In case of diagnostic simulations the
matrices of temperature and salinity have to be provided in the
files named |tempd| and |saltd| and data must be available for
the whole period of simulation.

If the |ibarcl|=1 option is used, the following additional forcing files 
must the provided:

\begin{itemize}
    \item A file with short wave solar radiation, relative humidity, 
		air temperature and cloudiness
    \item A file containing percipitation data
\end{itemize}

\begin{figure}[ht]
\begin{verbatim}
$para
        ibarcl = 1   temref = 18.    salref = 35.
$end
\end{verbatim}
\caption{Example of baroclinic simulation. The initial values for temperature
and salinity are set to 18 C and 35.}
\label{fig:baroc}
\end{figure}

