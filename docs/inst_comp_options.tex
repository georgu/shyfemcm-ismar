
%------------------------------------------------------------------------
%
%    Copyright (C) 1985-2020  Georg Umgiesser
%
%    This file is part of SHYFEM.
%
%    SHYFEM is free software: you can redistribute it and/or modify
%    it under the terms of the GNU General Public License as published by
%    the Free Software Foundation, either version 3 of the License, or
%    (at your option) any later version.
%
%    SHYFEM is distributed in the hope that it will be useful,
%    but WITHOUT ANY WARRANTY; without even the implied warranty of
%    MERCHANTABILITY or FITNESS FOR A PARTICULAR PURPOSE. See the
%    GNU General Public License for more details.
%
%    You should have received a copy of the GNU General Public License
%    along with SHYFEM. Please see the file COPYING in the main directory.
%    If not, see <http://www.gnu.org/licenses/>.
%
%    Contributions to this file can be found below in the revision log.
%
%------------------------------------------------------------------------
%%%%%%%%%%%%%%%%%%%%%%%%%%%%%%%%%%%%%%%%%%%
%former compilation.tex
In order to compile the model you will first have to adjust some settings
in the |Rules.make| file.
Note that every time that the file |Rules.make| is modified,
you will have to run the following command:
\begin{codem}
make cleanall
make
\end{codem}
Assuming that you are already in the SHYFEM
root directory (in our case it would be \ttt{\shydir} or
|/home/model/shyfemcm-ismar|), open the file
|Rules.make| with a text editor.
In the |Rules.make| the following options
can be set:

\begin{itemize}

\item |Compiler profile|. Set the level of verbosity of the messages. Use
|SPEED| if you want the maximum performances. Use the other options, in
case of errors, to have more informations.

\item |Compiler|. Set the compiler you want to use. Please see also
the section on needed software and the one on compatibility problems to
learn more about this choice. It is advisable to use the same type of
compiler for C and Fortran.


\item |Parallel compilation|. The code is parallelized either with OpenMP and MPI statements.
Here you can set if you want to use it or not.

\begin{itemize}
\item Running with OpenMP: \\
If you want to run with OMP, you have to change the Rules.make file as follows:

Set the compiler
\begin{codem}
 PARALLEL_OMP=true
\end{codem}

Then, in the str file, section para, you have to indicate the number of nodes/cores you want to use for parallel computation, e.g. |nomp = 8| means parallel on 8 cores.

\item Running with MPI: \\
|METIS| and the |MPI| library need to be installed (cf. section \ref{prereq})
The command |make test_mpi| can be used as supplementary check for
MPI installation - run it from the \shy{} root directory.
Note that it is suggested to install |METIS| via the apt command line
which usually installs the library in |/lib/x86\_64\_linux\_gnu| or |/usr/lib|.
You have to specify the path where the library is installed in the file Rules.make.
If you put |METIS| in |HOME/lib| with all the add-on libraries then \shy{} will
look automatically for it and you do not have to specify the directory in Rules.make. \\

To sum up these are the changes required in the file Rules.make to run \shy{} with the MPI option:
\begin{itemize}
 \item Set the compiler:
 \begin{codem}
  PARALLEL_MPI = NODE
 \end{codem}
 \item Set the partitioner:
 \begin{codem}
  PARTS = METIS
 \end{codem}
 \item Set the path where |METIS| is installed:
  \begin{codem}
  METISDIR = HOME/lib/metis
  \end{codem}
 or another path
\end{itemize}
%if METIS is installed in $HOME/lib$ (you should have a directory metis
%in this place) you can do everything above by just running
%
%	make rules$\_$mpi\\

To run shyfem in parallel with MPI, use:
\begin{codem}
mpirun -np 4 path_to_shyfem/shyfem str-file.str
\end{codem}
where np set the number of processes to run (above 4 are used) which corresponds
to the number of domain partitions.

\end{itemize}

\item |Solver for matrix solution|. There are four different solvers implemented.
The model need to solve a system of matrix equations. In order to do this, four different
solvers are implemented and can be set in the Rules.make file (flag SOLVER).

\subitem 1) The SPARSKIT solver is iterative and quite fast (default). It is a sequential solver but it can be used in parallel runs (the linear system solution will be calculated using only one processor);
\begin{codem}
SOLVER = SPARSKIT
\end{codem}
\subitem 2) The GAUSS solver is a robust direct solver, but quite slow;
\begin{codem}
SOLVER = GAUSS
\end{codem}
\subitem 3) The PARDISO solver is not included in the model code, but can be found in the Intel MKL and linked dynamically during the compilation. This solver is set as a direct solver but can be used also in an iterative mode.
\begin{codem}
SOLVER = PARDISO
\end{codem}
\subitem 4) The PETSC solver is a parallel solver with MPI.
Before running \shy{} with PETSC please be sure that the MPI mode (without PETSC)
run as expected (i.e. MPI has been sucessfully installed and simulations can be run in parallel) .
\begin{codem}
SOLVER = PETSC
\end{codem}

To use PETSC try to compile:
\begin{code}
	cd shyfem (or wherever your shyfem installation is)
	make cleanall
	make rules_std rules_petsc
	make fem
\end{code}

if you get a message that PETSC$\_$DIR is not in the library path run:\\

	export LD$\_$LIBRARY$\_$PATH=$\$$LD$\_$LIBRARY$\_$PATH:$\$$HOME/lib/petsc/lib \\

	this command should be inserted into .profile or .bashrc because this message will also appear every time you run shyfem\\

try to run shyfem:
\begin{code}
	mpirun -np 4 path_to_shyfem/shyfem str-file.str
\end{code}

\item |NetCDF library|. If you want output files in NetCDF format
you need the NetCDF library. Normally netcdf should already be installed. To enable netcdf you
must set in Rules.make the following:
\begin{codem}
NETCDF = true
NETCDFDIR = HOME/lib/netcdf
\end{codem}

If the directory where netcdf is not installed you have to give it
explicitly in NETCDFDIR.


\item |GOTM library|. The GOTM turbulence model (version 4.0.0) is already included in
the code. However, a newer and better tested version is available as an
external module. In order to use it please let this variable to true. This
is the recommended choice. You will need a Fortran 90 compiler to enable
this choice.
\begin{codem}
GOTM = true
\end{codem}

\item |Ecological module|. This option allows for the inclusion of an
ecological module into the code. Choices are between |EUTRO|, |AQUABC| and |BFM|. Please refer to information given in Section \ref{eco}
to run these programs. Set to |NONE| if no ecological module is present:
\begin{codem}
ECOLOGICAL = NONE
\end{codem}

\item |Wave module|. This option allows for the coupling between \shy{} and a wave module called WW3. Please specify in WW3DIR the base directory of WW3. The source code of WW3 that is suitable for the coupling with SHYFEM should be asked us by email at \ttt{georg.umgiesser@ismar.cnr.it}. If you may want to couple SHYFEM with WW3 set:
\begin{codem}
WW3 = true
WW3DIR = HOME/WW3
\end{codem}
Please refer to information given in Section \ref{ww3}
to compile and run WW3.

\end{itemize}

Once you have set all these options you can start compilation with

\begin{code}
    make clean
    make
\end{code}

This should compile everything. In case of compilation errors,
 messages will be shown both while the program is compiling as
well as at the bottom of the output (where a check occurs to see if
the main routines have been compiled).

Please remember that you will always have to run the commands above
when you change settings in the |Rules.make| file. If you only change
something in the code, or if you only change dimension parameters, it
might be enough to run only |make|, which only compiles the necessary
files. However, if you are in doubt, it is always a good idea to run
|make clean| or |make cleanall| before compiling, in order to start from
a clean state.
%%%%%%%%%%%%%%%%%%%%%%%%%%%%%%%%%%%%%%%
%former summary_installing.tex
\\
\\
A summary of administrative commands available in SHYFEM is given below: \vspace{0.5cm}

\begin{center}
\begin{tabular}{ l p{7cm} }
|make version|		&	shows version of distribution \\
|make clean|		&	deletes objects and executables from a previous
                        	compilation \\
|make cleanall|		&	same as |make clean| but also deletes
				compiled libraries \\
|make|		&	compiles SHYFEM \\
%|make doc|		&	makes this manual (|femdoc/shyfem.pdf|) \\
|make check_software|	&	checks the availability of installed software \\
|make check_compilation|&	checks if all programs have been compiled \\
|make changed|		&	finds files that are changed with respect to the
				original distribution \\
|make changed_zip|	&	zips files that are changed with respect to the
				original distribution to the file
				|changed_zip.zip| \\
|make install|		&	installs SHYFEM \\
|make uninstall|	&	uninstalls SHYFEM \\
\end{tabular}
\end{center}

\vspace{0.5cm}
Finally, if you have installed the model with |make install|,
the following utility commands are available \vspace{0.5cm}

\begin{center}
\begin{tabular}{ l l }
|shyfemdir|		&	shows information about actual SHYFEM
				settings \\
|shyfemdir fem_dir|	&	sets |fem_dir| to be the new default
				SHYFEM version \\
|shyfeminstall|		&	shows information about original SHYFEM
				installation \\
|shyfemcd|		&	moves into root of actual SHYFEM directory \\
|shypath|		&	shows information about the actual path \\
                        &       (executive directory of SHYFEM, the bin directory of SHYFEM, \\
                        &        the name of the SHYFEM directory ...) \\
\end{tabular}
\end{center}


