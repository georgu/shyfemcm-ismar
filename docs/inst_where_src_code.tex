
%------------------------------------------------------------------------
%
%    Copyright (C) 1985-2020  Georg Umgiesser
%
%    This file is part of SHYFEM.
%
%    SHYFEM is free software: you can redistribute it and/or modify
%    it under the terms of the GNU General Public License as published by
%    the Free Software Foundation, either version 3 of the License, or
%    (at your option) any later version.
%
%    SHYFEM is distributed in the hope that it will be useful,
%    but WITHOUT ANY WARRANTY; without even the implied warranty of
%    MERCHANTABILITY or FITNESS FOR A PARTICULAR PURPOSE. See the
%    GNU General Public License for more details.
%
%    You should have received a copy of the GNU General Public License
%    along with SHYFEM. Please see the file COPYING in the main directory.
%    If not, see <http://www.gnu.org/licenses/>.
%
%    Contributions to this file can be found below in the revision log.
%
%------------------------------------------------------------------------

The source code of the model is hosted at the following github adress: \\
https://github.com/georgu/shyfemcm-ismar .

To get the code there are two options:
\begin{enumerate}
   \item clone the git repository of the code through the command:
\begin{code}
 git clone https://github.com/georgu/shyfemcm-ismar.git
\end{code}
This choice implies that you already have git installed on your computer.
You can check it running:
\begin{code}
 git --version
\end{code}
If it's not installed, a version can be installed through the package manager
 of your Linux distribution.
The git clone command above will automatically create a directory called
shyfemcm-ismar containing the \shy{} model. If you wish to store the model
in a different directory, use:
\begin{codem}
 git clone https://github.com/georgu/shyfemcm-ismar.git my_directory
\end{codem}
Now |/home/model/my_directory| directory will be the root of the SHYFEM model.
If you wish to install the latest version, run the following commands after cloning:
\begin{codem}
cd my_directory
git checkout VERS_X_X_X
\end{codem}
The number of the latest tagged version can be found here:
https://github.com/georgu/shyfemcm-ismar/tags .
All the developments done at the ISMAR institute are stored in this
repository~(https://github.com/georgu/shyfemcm-ismar) where you can find
several branches. The repository organization is summed up here %but more details to use git with \shy{} can be found in the SHYFEM GIT GUIDE (\textcolor{red}{insert ref, soon pushed in github}). \\
%The structure repository is organized as follows:
\begin{itemize}
\item
main:		this is the main branch which mainly contains the
		stable versions of the code
\item
develop:	this branch contains all the new developments of ISMAR
		derived from other branches.
		If you need a special feature that has not been yet
		pushed in the main branch, please look here.
\item
feature1:	these is a feature branch. Here some new features
		will be implemented. Once the feature is tested
		it will be transferred to the develop branch.
\item
feature2:	etc...
\end{itemize}
The main advantage of this choice is that you can easily keep your model
updated according to the changes done by the developers (bug fixes, new features...).
   \item download a compressed archive of the latest stable version.
Stable versions are tagged versions which can be found at
https://github.com/georgu/shyfemcm-ismar/tags . \\
Once you have downloaded the model distribution, move the file to
the directory in which you want to install the model and unpack the
distribution. In the following we will assume that the file is in your
home directory and your home directory is called \ttt{\basedir}. However,
any other directory works as well. To unpack the distribution
in your home directory, move there and run the command:

\begin{codem}
    cd \basedir
    tar xzvf \shydist
\end{codem}

At this point a new folder named \ttt{\shydir} has been created:
	this directory is the root of the \shy{} model.
\end{enumerate}


%%\item compile the model
%%
%%	go into the directory that has been created and run
%%		make\\
%%	this should compile the model with standard flags\\
%%	you can always run "make help" to see other targets\\
%%	running make check$\_$software will tell you what software is missing for shyfem to be installed
%%
%%\item make a symbolic link:
%%this is not necessary, but useful\\
%%	goto your home directory and run\\
%%		ln -s dir-where-shyfem-is-installed shyfem\\
%%	this will create a link to the new shyfem directory\\
%%	be sure you do not have a shyfem directory (or link) in your home
%%
%%\item update the model
%%
%%	this is needed when new functionality has been added to the model with these commands you can download the latest version easily\\
%%
%%	updating only works if you have git installed on your computer and you have chosen option 2a above. In case you have chosen 	option 2b you will have to download the model again from scratch.\\
%%
%%	goto the shyfem directory where the model is installed\\
%%		git fetch\\
%%		git pull\\
%%		make\\
%%
%%	this will get the latest model version and compile it\\
%%
%%	if you get an error with "git pull" you probably have changed something in the model code and the pull would overwrite your changes. Save your changes somewhere, then do\\
%%		git checkout file(s)\\
%%	where file(s) are the offending file(s)\\
%%		git pull\\
%%	and then compare the difference between your files and the new ones\\



%%%%%%%%%%%%%%%%%%%%%%%%%%%%%%%%%%%%%%%%%%







