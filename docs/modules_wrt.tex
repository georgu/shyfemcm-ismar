
%------------------------------------------------------------------------
%
%    Copyright (C) 1985-2020  Georg Umgiesser
%
%    This file is part of SHYFEM.
%
%    SHYFEM is free software: you can redistribute it and/or modify
%    it under the terms of the GNU General Public License as published by
%    the Free Software Foundation, either version 3 of the License, or
%    (at your option) any later version.
%
%    SHYFEM is distributed in the hope that it will be useful,
%    but WITHOUT ANY WARRANTY; without even the implied warranty of
%    MERCHANTABILITY or FITNESS FOR A PARTICULAR PURPOSE. See the
%    GNU General Public License for more details.
%
%    You should have received a copy of the GNU General Public License
%    along with SHYFEM. Please see the file COPYING in the main directory.
%    If not, see <http://www.gnu.org/licenses/>.
%
%    Contributions to this file can be found below in the revision log.
%
%------------------------------------------------------------------------

SHYFEM includes a module for the computation of the water renewal time (WRT) of semi-enclosed basins. 
Following the method proposed by \cite{Cucco2006}, WRT is defined as the time required for 
each node of the domain to replace the mass of a conservative tracer initially released into the basin with 
new water. It is estimated by computing the advection and diffusion of a conservative tracer initially released 
within a selected domain and by calculating the integral of the tracer local remnant function at the end of 
the simulation run. Assuming an exponential decay of the tracer initial concentration corresponds to its half-life. The computation can be carried out for the entire domain or for a selected sub-area and for 
a specific time interval. 
The main parameters to be set are the time of the initial tracer release (|itmin|), the 
code of elements out of the selected sub-area (|iaout|) and the tracer initial concentration (|c0|). Additional 
features are:
\begin{itemize} 
\item |iret|, the possibility of neglecting the tracer return flow
\item |bstir|, the simulation of a completely stirred domain by replacing at each time step the local concentration with the basin average value
\item |percmin|, the definition of the minimal percentage of the initial tracer concentration to end the WRT computation
\item |idtwrt|,  the possibility of sequential WRT computation by defining the time step to reset the tracer concentration to its initial value
\item |blog|, the possibility of using logarithmic regression to compute the WRT therefore assuming an exponential time decay of the tracer concentration.
\end{itemize}