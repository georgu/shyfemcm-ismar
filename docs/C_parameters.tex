The parameter input file has the extension .str and is composed of different sections, below described. For each section the list of setting parameters that can be declared are presented.
\subsection{Section {\tt \$title}}
\label{parameter}

This section must be always the first section in the parameter input file.
It contains only three lines. An example is given in
\Fig\ref{fig:titleexample}.

\begin{figure}[hbtp]
\begin{verbatim}
$title
        free one line description of simulation
        name_of_simulation
        name_of_basin
$end
\end{verbatim}
\caption{Example of section {\tt \$title}}
\label{fig:titleexample}
\end{figure}

The first line of this section is a free one line description of
the simulation that is to be carried out. The next line contains
the name of the simulation.
All created files will use this name in the main part of the file name
with different extensions. Therefore the hydrodynamic output file
(extension |out|) will be named |name_of_simulation.out|.
The last line gives the name of the basin file to be used. This
is the pre-processed file of the basin with extension |bas|.
In our example the basin file |name_of_basin.bas| is used.

The directory where this files are read from or written to depends
on the settings in section {\tt \$name}. Using the default
the program will read from and write to the current directory.

\subsection{Section {\tt \$para}}
\paragraph{Compulsory time parameters}

This parameters are compulsory parameters that define the
period of the simulation. They must be present in all cases.

\descrpitem{|itanf|}
\descrptext{%
Start of simulation. (Default 0)
}
\par
\descrpitem{|itend|}
\descrptext{%
End of simulation.
}
\par
\descrpitem{|idt|}
\descrptext{%
Time step of integration.
}
\par


\paragraph{Output parameters}

The following parameters deal with the output frequency
and start time to write external files. The content of the various
output files should be looked up in the appropriate section.

If no output time step is decleared, the output file will not be written.
The default value of the output time step is 0., so, if not decleared, there will be no output file. If the time step of the
output files is equal to the time step of the simulation then
at every time step the output file is written. The default start time
of the output is |itanf|, the start of the simulation.

\descrpitem{|idtout|\\|itmout|}
\descrptext{%
Time step and start time for writing to file OUT,
the file containing the general hydrodynamic results.
}
\par

\descrpitem{|idtext|\\|itmext|}
\descrptext{%
Time step and start time for writing to file EXT,
the file containing hydrodynamic data of extra points.
The extra points for which the data is written
to this file are given in section |extra| of
the parameter file.
}
\par

\descrpitem{|idtrst|}
\descrptext{%
Time step for writing the restart
file (extension RST). No restart file is written
with |idtrst| equal to 0. A negative value
is also possible for the time step. In this case
the time step used is |-idtrst|, but the file is
overwritten every time. It therefore contains
always only the last written restart record. The
special value of |idtrst = -1| will write only the
last time step of the simulation in the restart file.
This is useful if you want to start another
simulation from the last output. (Default 0)
}
\par
\descrpitem{|itmrst|}
\descrptext{%
Start time for writing the restart file. If
not given it is the beginning of the simulation.
}
\par
\descrpitem{|itrst|}
\descrptext{%
Time to use for the restart. If a restart
is performed, then the file name containing
the restart data has to be specified in |restrt|
and the time record corresponding to |itrst|
is used in this file. A value of -1 is also possible.
In this case the last record in the restart file
is used for the restart and the simulation starts
from this time. Be aware that this option changes
the parameter |itanf| to the time of the last
record found in |restrt|.
}
\par
\descrpitem{|ityrst|}
\descrptext{%
Type of restart. If 0 and the restart file is not
found the program will exit with an error. Otherwise
the program will simply continue with a cold start.
If |ityrst| is 1 and the given time record is not
found in the file it will exit with error. If
it is 2 it will initialize all values from the
first time record after |itrst|. Therefore, the
value of 2 will guarantee that the program will not
abort and continue running, but it might not
be doing what you intended. (Default 0)
}
\par
\descrpitem{|flgrst|}
\descrptext{%
This variable indicates which variables are read
from the restart file. By default all available
variables are read and used. If some variables
are not wanted (because, e.g., you want to restart
from a different T/S field), this fact can be indicated
in |flgrst|. 1 indicates restart of hydro values,
10 the depth values, 100 T/S values, 1000 the tracer
concentration, 10000 vertical velocities and
100000 the ecological variables. Therefore, a value
of 10111 indicates a restart of everything except
the tracer and the ecological values. The default
value for |flgrst| is -1, which means 111111.
}
\par

\descrpitem{|idtres|\\|itmres|}
\descrptext{%
Time step and start time for writing to file RES,
the file containing residual hydrodynamic data.
}
\par

\descrpitem{|idtrms|\\|itmrms|}
\descrptext{%
Time step and start time for writing to file RMS,
the file containing hydrodynamic data of root mean
square velocities.
}
\par

\descrpitem{|idtflx|\\|itmflx|}
\descrptext{%
Time step and start time for writing to file FLX,
the file containing discharge data through defined
sections.
The transects for which the discharges are computed
are given in section |flux| of
the parameter file.
}
\par

\descrpitem{|idtvol|\\|itmvol|}
\descrptext{%
Time step and start time for writing to file VOL,
the file containing volume information of areas
defined by transects.
The transects that are used to compute the volumes
are given in section |volume| of
the parameter file.
}
\par

\descrpitem{|netcdf|}
\descrptext{%
This parameter chooses output in NetCDF format
if |netcdf| is 1, else the format is unformatted
fortran files. (Default 0)
}
\par

\descrpitem{|idtoff|}
\descrptext{%
handles offline mode (default 0):
\begin{description}
\item[0] do nothing (no offline routines called)
\item[$>$0] write offline data file (.off) with time step |idtoff|.
\item[$<$0] reads offline data from file |offlin|
defined in section |name|. Usage:
\begin{description}
\item[-1] uses offline hydro results
\item[-2] uses offline T/S results
\item[-4] uses offline turbulence results
\end{description}
\end{description}
}
\par

\descrpitem{|itmoff|}
\descrptext{%
Start time for writing to file OFF,
the file containing data for offline runs.
}
\par


\paragraph{General time and date parameters}

A time and date can be assigned to the simulation. These values
refer to the time 0 of the FEM model. The format for the date is
YYYYMMDD and for the time HHMMSS.
You can also give a time zone if your time is not referring to
GMT but to another time zone such as MET.

|date|                The real date corresponding to time 0. (Default 0)
|time|                The real time corresponding to time 0. (Default 0)
|tz|                  The time zone you are in. This is 0 for GMT, 1 for MET
and 2 for MEST (MET summer time). (Default 0)


\paragraph{Model parameters}

The next parameters define the inclusion or exclusion of
certain terms of the primitive equations.

\descrpitem{|ilin|}
\descrptext{%
Linearization of the momentum equations. If |ilin|
is different from 0 the advective terms are not
included in the computation. (Default 1)
}
\par
\descrpitem{|itlin|}
\descrptext{%
This parameter decides how the advective (non-linear)
terms are computed. The value of 0 (default) uses
the usual finite element discretization over a single
element. The value of 1 chooses a semi-lagrangian
approach that is theoretically stable also for
Courant numbers higher than 1. It is however recommended
that the time step is limited using |itsplt| and
|coumax| described below. (Default 0)
}
\par
\descrpitem{|iclin|}
\descrptext{%
Linearization of the continuity equation. If |iclin|
is different from 0 the depth term in the continuity
equation is taken to be constant. (Default 0)
}
\par

The next parameters allow for a variable time step in the
hydrodynamic computations. This is especially important for the
non-linear model (|ilin=0|) as in this case the criterion
for stability cannot be determined a priori and in any case the
time integration will not be unconditionally stable.

The variable time steps allows for longer basic time steps
(here called macro time steps) which have to be set in |idt|.
It then computes the optimal time step (here micro time step)
in order to not exceed the given Courant number. However,
the value for the macro time step will never be exceeded.

Normally time steps are always given in full seconds. This is still
true when specifying the macro time step |idt|.
In older versions, the computed micro time steps also had to be
 full integers. Starting from version 7.1, fractional time steps
were allowed. This allows time steps smaller than 1s.

\descrpitem{|itsplt|}
\descrptext{%
Type of variable time step computation. If this value
is 0, the time step will be kept constant at its initial
value. A value of 1 divides the initial time step into
(possibly) equal parts, but makes sure that at the end
of the micro time steps one complete macro time
step has been executed. The mode |itsplt| = 2
ignores the macro time step, but always
uses the biggest time step possible. In this case,
it is not assured that after some micro time steps
a macro time step will be recovered. Please note
that the initial macro time step will never be exceeded.
In any case, the time step will always be rounded to the
next lower integer value. This is not the case with
|itsplt| = 3 where the highest possible fractional time step
will be used. (Default 0)
}
\par
\descrpitem{|coumax|}
\descrptext{%
Normally the time step is computed in order to not
exceed the Courant number of 1. However, in some cases
the non-linear terms are stable even for a value higher
than 1 or there is a need to achieve a lower Courant number.
Setting |coumax| to the desired Courant number
achieves exactly this effect. (Default 1)
}
\par
\descrpitem{|idtsyn|}
\descrptext{%
In case of |itsplt| = 2 this parameter makes sure that
after a time of |idtsyn| the time step will be syncronized
to this time. Therefore, setting |idtsyn| = 3600 means
that there will be a time stamp every hour, even if the model
has to take one very small time step in order to reach that
time. This parameter is useful
only for |itsplt| = 2 and its default value of
0 does not make any syncronization.
}
\par
\descrpitem{|idtmin|}
\descrptext{%
This variable defines the smallest time step possible
when time step splitting is enabled. Normally the smallest
time step is 1 second. Please set |idtmin| to values
smaller than 1 in order to allow for fractional time steps.
A value of 0.001 allows for timesteps of down to
1 millisecond. (Deault 1)
}
\par

These parameters define the weighting of time level in the
semi-implicit algorithm. With these parameters the damping
of gravity or Rossby waves can be controlled. Only modify them if
you know what you are doing.

\descrpitem{|azpar|}
\descrptext{%
Weighting of the new time level of the transport
terms in the continuity equation. (Default 0.5)
}
\par
\descrpitem{|ampar|}
\descrptext{%
Weighting of the new time level of the pressure
term in the momentum equations. (Default 0.5)
}
\par
\descrpitem{|afpar|}
\descrptext{%
Weighting of the new time level of the Coriolis
term in the momentum equations. (Default 0.5)
}
\par
\descrpitem{|avpar|}
\descrptext{%
Weighting of the new time level of the non-linear
advective terms in the momentum equations. (Default 0.0)
}
\par

The next parameters define the weighting of time level for the
vertical stress and advection terms. They guarantee the stability
of the vertical system. For this reason they are normally set to
1 which corresponds to a fully implicit discretization. Only
modify them if you know what you are doing.

\descrpitem{|atpar|}
\descrptext{%
Weighting of the new time level of the vertical
viscosity in the momentum equation. (Default 1.0)
}
\par
\descrpitem{|adpar|}
\descrptext{%
Weighting of the new time level of the vertical
diffusion in the scalar equations. (Default 1.0)
}
\par
\descrpitem{|aapar|}
\descrptext{%
Weighting of the new time level of the vertical
advection in the scalar equations. (Default 1.0)
}
\par


\paragraph{Coriolis parameters}

The next parameters define the parameters to be used
with the Coriolis terms.

\descrpitem{|icor|}
\descrptext{%
If this parameter is 0, the Coriolis terms are
not included in the computation. A value of 1
uses a beta-plane approximation with a variable
Coriolis parameter $f$, whereas a value of
2 uses an f-plane approximation where the
Coriolis parameter $f$ is kept constant over the
whole domain. (Default 0)
}
\par
\descrpitem{|dlat|}
\descrptext{%
Average latitude of the basin. This is used to
compute the Coriolis parameter $f$. This parameter
is not used if spherical coordinates are used
(|isphe|=1) or if a coordinate 	projection is set
(|iproj| $>$0). (Default 0)
}
\par
\descrpitem{|isphe|}
\descrptext{%
If 0 a cartesian coordinate system is used,
if 1 the coordinates are in the spherical system (lat/lon).
Please note that in case of spherical coordinates the
Coriolis term is always included in the computation, even
with |icor| = 0. If you really do not want to use the
Coriolis term, then please set |icor| = -1. The default is
-1, which means that the type of coordinate system will
be determined automatically.
}
\par


\paragraph{Depth parameters}

The next parameters deal with handling depth values of the basin.

\descrpitem{|href|}
\descrptext{%
Reference depth. If the depth values of the basin and
the water levels are referred to mean sea level, |href|
should be 0 (default value). Else this value is
subtracted from the given depth values. For example,
if |href = 0.20| then a depth value in the basin
of 1 meter will be reduced to 80 centimeters.
}
\par

\descrpitem{|hzmin|}
\descrptext{%
Minimum total water depth that will remain in a
node if the element becomes dry. (Default 0.01 m)
}
\par
\descrpitem{|hzoff|}
\descrptext{%
Total water depth at which an element will be
taken out of the computation because it becomes dry.
(Default 0.05 m)
}
\par
\descrpitem{|hzon|}
\descrptext{%
Total water depth at which a dry element will be
re-inserted into the computation.
(Default 0.10 m)
}
\par

\descrpitem{|hmin|}
\descrptext{%
Minimum water depth (most shallow) for the whole
basin. All depth values of the basin will be adjusted
so that no water depth is shallower than |hmin|.
(Default is no adjustment)
}
\par
\descrpitem{|hmax|}
\descrptext{%
Maximum water depth (deepest) for the whole
basin. All depth values of the basin will be adjusted
so that no water depth is deeper than |hmax|.
(Default is no adjustment)
}
\par

\paragraph{Bottom friction}

The friction term in the momentum equations can be written as
$Ru$ and $Rv$ where $R$ is the variable friction coefficient and
$u,v$ are the velocities in $x,y$ direction respectively.
The form of $R$ can be specified in various ways. The value of
|ireib| is chosen between the formulations. In the parameter
input file a value $\lambda$ is specified that is used in
the formulas below. In a 2D simulation the Strickler (2) or the Chezy (3)
formulation is the preferred option, while for a 3D simulation is
recommended to use the drag coefficient (5) or the roughness length
formulation (6).

\descrpitem{|ireib|}
\descrptext{%
Type of friction used (default 0):
\begin{description}
\item[0] No friction used
\item[1] $R=\lambda$ is constant
\item[2] $\lambda$ is the Strickler coefficient.
In this formulation $R$ is written as
$R = \frac{g}{C^2} \frac{\vert u \vert}{H}$
with $C=k_s H^{1/6}$ and $\lambda=k_s$ is
the Strickler coefficient. In the above
formula $g$ is the gravitational acceleration,
$\vert u \vert$ the modulus of the current velocity
and $H$ the total water depth.
\item[3] $\lambda$ is the Chezy coefficient.
In this formulation $R$ is written as
$R = \frac{g}{C^2} \frac{\vert u \vert}{H}$
and $\lambda=C$ is the Chezy coefficient.
\item[4] $R=\lambda/H$ with $H$ the total water depth
\item[5] $\lambda$ is a constant drag coefficient and $R$ is
computed as $R=\lambda\frac{\vert u \vert}{H}$
\item[6] $\lambda$ is the bottom roughness length and $R$ is
computed through the formula
$R=C\frac{\vert u \vert}{H}$ with
$C=\big(\frac{0.4}{log(\frac{\lambda+0.5H}
{\lambda})}\big)^2$
\item[7] If $\lambda \geq 1$ it specifies the Strickler
coefficient (|ireib=2|), otherwise it specifies a
constant drag coefficient (|ireib=5|).
\end{description}
}
\par
\descrpitem{|czdef|}
\descrptext{%
The default value for the friction parameter $\lambda$.
Depending on the value of |ireib| the coefficient $\lambda$
is representing linear friction, a constant drag coefficient,
the Chezy or Strickler parameter, or the roughness length
(default 0).
}
\par
\descrpitem{|iczv|}
\descrptext{%
Normally the bottom friction coefficient
(such as Strickler, Chezy, etc.)
is evaluated at every time step (|iczv| = 1).
If for some reason this behavior is not desirable,
|iczv| = 0 evaluates this value only before the
first time step, keeping it constant for the
rest of the simulation. Please note that this is
only relevant if you have given more than one bottom
friction value (inflow/outflow) for an area. The
final value of $R$ is computed at every time step
anyway. (default 1)
}
\par

The value of $\lambda$ may be specified for the whole basin through
the value of |czdef|. For more control over the friction parameter
it can be also specified in section |area| where the friction
parameter depending on the type of the element may be varied. Please
see the paragraph on section |area| for more information.

\paragraph{Physical parameters}

The next parameters describe physical values that can be adjusted
if needed.

\descrpitem{|rowass|}
\descrptext{%
Average density of sea water. (Default 1025 \densityunit)
}
\par
\descrpitem{|roluft|}
\descrptext{%
Average density of air. (Default 1.225 \densityunit)
}
\par
\descrpitem{|grav|}
\descrptext{%
Gravitational acceleration. (Default 9.81 \accelunit)
}
\par

\paragraph{Wind parameters}

The next two parameters deal with the wind stress to be
prescribed at the surface of the basin.

The wind data can either be specified in an external file (ASCII
or binary) or directly in the parameter file in section |wind|.
The ASCII file or the wind section contain three columns, the first
giving the time in seconds, and the others the components of
the wind speed. Please see below how the last two columns are
interpreted depending on the value of |iwtype|. For the format
of the binary file please see the relative section.
If both a wind file and section |wind| are given, data from the
file is used.

The wind stress is normally computed with the following formula
\beq
\tau^x = \rho_a c_D \vert u \vert u^x \quad
\tau^y = \rho_a c_D \vert u \vert u^y
\eeq
where $\rho_a,\rho_0$ is the density of air and water respectively,
$u$ the modulus of wind speed and $u^x,u^y$ the components of
wind speed in $x,y$ direction. In this formulation $c_D$ is a
dimensionless drag coefficient that varies between 1.5 \ten{-3} and
3.2 \ten{-3}. The wind speed is normally the wind speed measured
at a height of 10 m.

|iwtype|      The type of wind data given (default 1):
\begin{description}
\item[0] No wind data is processed
\item[1] The components of the wind is given in [m/s]
\item[2] The stress ($\tau^x,\tau^y$) is directly specified
\item[3] The wind is given in speed [m/s] and direction
[degrees]. A direction of 0\degrees{} specifies
a wind from the north, 90\degrees{} a wind
from the east etc.
\item[4] As in 3 but the speed is given in knots
\end{description}
\descrpitem{|itdrag|}
\descrptext{%
Formula to compute the drag coefficient.
\begin{description}
\item[0] constant value given in |dragco|.
\item[1] Smith and Banke (1975) formula
\item[2] Large and Pond (1981) formula
\item[3] Spatio/temporally varing in function of wave. Need
the coupling with WWMIII.
\item[4] Spatial/temporal varying in function of heat flux. Only for iheat = 6 and iheat = 8.
\end{description}
(Default 0)
}
\par
|dragco|      Drag coefficient used in the above formula.
Please note that in case
of |iwtype| = 2 this value is of no interest, since the
stress is specified directly. (Default 2.5E-3)
|wsmax|       Maximum wind speed allowed in [m/s]. This is in order to avoid
errors if the wind data is given in a different format
from the one specified by |iwtype|. (Default 50)

\descrpitem{|wslim|}
\descrptext{%
Limit maximum wind speed to this value [m/s]. This provides
an easy way to exclude strong wind gusts that might
blow up the simulation. Use with caution.
(Default -1, no limitation)
}
\par

\paragraph{Meteo and heat flux parameters}
The next parameters deal with the heat and meteo forcing.

\descrpitem{|iheat|}
\descrptext{%
The type of heat flux algorithm (Default 1):
\begin{description}
\item[1] As in the AREG model
\item[2] As in the POM model
\item[3] Following A. Gill
\item[4] Following Dejak
\item[5] As in the GOTM model
\item[6] Using the COARE3.0 module
\item[7] Read  sensible, latent and longwave fluxes from file
\item[8] Heat flux with bulk-formulae by \cite{Pettenuzzo2010}, computing Net Long wave radiation \cite{Bignami1995}, Sensible heat \cite{Kondo1975}, Latent heat \cite{Kondo1975}, Evaporation \cite{Kondo1975}, Short Wave Solar Radiation \cite{Reed1977}. Selecting iheat = 8 you need to activate also itdrag=4 since the spatial/temporal-varying wind drag coefficient is computed with \cite{Hellermann1983} parametrization, as function of heat flux.
\end{description}
}
\par
\descrpitem{isolp}
\descrptext{%
The type of solar penetration parameterization by one or more exponential decay curves. isolp = 0 sets an e- folding decay of radiation (one exponential decay curve) as function of depth hdecay. isolp = 1 sets a profile of solar radiation with two length scale of penetration. Following the \cite{Jerlov1968} classification the type of water is clear water (type I). (Default 0)
}
\par

\descrpitem{iwtyp}
\descrptext{%
The water types from clear water (type I) to the most turbid water (coastal water 9) following the classification of Jerlov (1968). From clear water type I to type III data are taken from \cite{Paulson1977}. Water type from 1 (option n. 5) to type 9 (option n. 9) were retrieved interpolating Jerlov data so they are still experimental. The possible values for iwtyp are:
\begin{description}
\item[0] clear water type I
\item[1] type IA
\item[2] type IB
\item[3] type II
\item[4] type III
\item[5] type 1
\item[6] type 3
\item[7] type 5
\item[8] type 7
\item[9] type 9
\end{description}
}
\par

\descrpitem{|hdecay|}
\descrptext{%
Depth of e-folding decay of radiation [m]. If |hdecay| = 0
everything is absorbed in first layer (Default 0).
}
\par

\descrpitem{|botabs|}
\descrptext{%
Heat absorption at bottom [fraction] (Default 0).
\begin{description}
\item[=0] everything is absorbed in last layer
\item[=1] bottom absorbs remaining radiation
\end{description}
}
\par

\descrpitem{|albedo|}
\descrptext{%
General albedo (Default 0.06).
}
\par

\descrpitem{|albed4|}
\descrptext{%
Albedo for temp below 4 degrees (Default 0.06).
}
\par

\descrpitem{|imreg|}
\descrptext{%
Regular meteo data (Default 0).
}
\par

\descrpitem{|ievap|}
\descrptext{%
Compute evaporation mass flux (Default 0).
}
\par


\paragraph{Parameters for 3d}
The next parameters deal with the layer structure in 3D.

\descrpitem{|dzreg|}
\descrptext{%
Normally the bottom of the various layers are given in
section |\$levels|. If only a regular vertical grid is desired
then the parameter |dzreg| can be used. It specifies the spacing
of the vertical layers in meters. (Default is 0, which means
that the layers are specified explicitly in |\$levels|.
}
\par

The last layer (bottom layer) is treated in a special way. Depending on
the parameter |ilytyp| there are various cases to be considered. A value
of 0 leaves the last layer as it is, even if the thickness is very small.
A value of 1 will always eliminate the last layer, if it has not full
layer thickness. A value of 2 will do the same, but only if the last layer
is smaller than |hlvmin| (in units of fraction). Finally, a value of
3 will add the last layer to the layer above, if its layer thickness
is smaller than |hlvmin|.

\descrpitem{|ilytyp|}
\descrptext{%
Treatment of last (bottom) layer. 0 means no adjustment,
1 deletes the last layer, if it is not a full layer,
2 only deletes it
if the layer thickness is less than |hlvmin|, and 3
adds the layer thickness to the layer above if it is smaller
than |hlvmin|. Therefore, 1 and 2 might change the
total depth and layer structure, while 3 only might
change the layer structure. The value of 1 will always
give you full layers at the bottom.
}
\par
\descrpitem{|hlvmin|}
\descrptext{%
Minimum layer thickness for last (bottom) layer used when
|ilytyp| is 2 or 3. The unit is fractions of the nominal
layer thickness. Therefore, a value of 0.5 indicates that
the last layer should be at least half of the full
layer.
}
\par

 With $z-$layers the treatment of the free-surface must be addressed.
What happen if the water level falls below the first $z$-level?
A $z-$star type vertical grid deformation can be deployed.
The next parameter specify the number of surface layers that are moving.
\par
\descrpitem{|nzadapt|}
\descrptext{%
Parameter that controls the number of surface $z-$layers that are moving.
The value $|nzadapt|\le 1$ corresponds to standard $z-$layers (Default).
Then, some care is needed to define the first interface sufficiently
deep to avoid the well-known "drying" of the first layer.
The value of $|nzadapt| = N_{tot}$, with $N_{tot}$ the total number
of $z$-layers, is $z-$star (all layers are moving).
Other values of $1<|nzadapt|<N_{tot}$ corresponds to move, at minimum,
the first |nzadapt| surface layers with $z-$star.
These feature is still experimental.
}
\par

The above parameters are dealing with zeta layers, where every layer
has constant thickness, except the surface layer which is varying with
the water level. The next parameters deal with sigma layers where all
layers have varying thickness with the water level.
\par
\descrpitem{|nsigma|}
\descrptext{%
Number of sigma layers for the run. This parameter can
be given in alternative to specifying the sigma layers
in |\$levels|. Only regularly spaced sigma levels
will be created.
}
\par

\par
\descrpitem{|hsigma|}
\descrptext{%
This is still an experimental feature. It allows to use
sigma layers above zeta layers. |hsigma| is the depth where
the transition between these two types of layers
is occurring.
}
\par

The next parameters deal with vertical diffusivity and viscosity.

\descrpitem{|diftur|}
\descrptext{%
Vertical turbulent diffusion parameter for the tracer.
(Default 0)
}
\par
\descrpitem{|difmol|}
\descrptext{%
Vertical molecular diffusion parameter for the tracer.
(Default 1.0e-06)
}
\par
\descrpitem{|vistur|}
\descrptext{%
Vertical turbulent viscosity parameter for the momentum.
(Default 0)
}
\par
\descrpitem{|vismol|}
\descrptext{%
Vertical molecular viscosity parameter for the momentum.
(Default 1.0e-06)
}
\par

\descrpitem{|dhpar|}
\descrptext{%
Horizontal diffusion parameter (general).
(Default 0)
}
\par

The next parameters deal with the control of the scalar transport
and diffusion equation. You have the possibility to prescribe the tvd scheme
desired and to limit the Courant number.

\descrpitem{|itvd|}
\descrptext{%
Type of the horizontal advection scheme used for
the transport and diffusion
equation. Normally an upwind scheme is used (0), but setting
the parameter |itvd| to a value greater than 0
chooses a TVD scheme. A value of 1 will use a TVD scheme
based on the average gradient, and a value of 2 will use
the gradient of the upwind node (recommended).
This feature
is still experimental, so use with care. (Default 0)
}
\par
\descrpitem{|itvdv|}
\descrptext{%
Type of the vertical advection scheme used for
the transport and diffusion
equation. Normally an upwind scheme is used (0), but setting
the parameter |itvd| to 1 chooses a TVD scheme. This feature
is still experimental, so use with care. (Default 0)
}
\par
\descrpitem{|rstol|}
\descrptext{%
Normally the internal time step for scalar advection is
automatically adjusted to produce a Courant number of 1
(marginal stability). You can set |rstol| to a smaller value
if you think there are stability problems. (Default 1)
}
\par


\paragraph{Various parameters}

The next parameters refer to various groups not described previously.

\descrpitem{|tauvel|}
\descrptext{%
If you have velocity observations given in file
|surfvel| then you can specify the relaxation
parameter $\tau$ in the variable |tauvel|. (Default 0,
which means no assimilation of velocities)
}
\par

\descrpitem{|rtide|}
\descrptext{%
If |rtide| = 1 the model calculates equilibrium tidal
potential and load tides and uses these to force the
free surface (Default 0).
}
\par


\paragraph{Temperature and salinity}

The next parameters deal with the transport and diffusion
of temperature and salinity. Please note that in order to compute
T/S, the parameter |ibarcl| must be different from 0. In this case
T/S advection is computed, but may be selectively turned off setting
one of the two parameters |itemp| or |isalt| explicitly to 0.

\descrpitem{|itemp|}
\descrptext{%
Flag if the temperature computation is done.
A value different from 0 computes the temperature transport
and diffusion. (Default 1)
}
\par
\descrpitem{|isalt|}
\descrptext{%
Flag if the salinity computation is done.
A value different from 0 computes the salinity transport
and diffusion. (Default 1)
}
\par

The next parameters set the initial conditions for temperature and salinity.
Both the average value and the stratification can be specified.

\descrpitem{|temref|}
\descrptext{%
Reference (initial) temperature of the water in
centigrade. (Default 0)
}
\par
\descrpitem{|salref|}
\descrptext{%
Reference (initial) salinity of the water in
psu (practical salinity units) or ppt.
(Default 0)
}
\par
\descrpitem{|tstrat|}
\descrptext{%
Initial temperature stratification in units of [C/km].
A positive value indicates a stable stratification.
(Default 0)
}
\par
\descrpitem{|sstrat|}
\descrptext{%
Initial salinity stratification in units of [psu/km].
A positive value indicates a stable stratification.
(Default 0)
}
\par

The next parameters deal with horizontal diffusion of temperature
and salinity. These parameters overwrite the general parameter for
horizontal diffusion |dhpar|.

\descrpitem{|thpar|}
\descrptext{%
Horizontal diffusion parameter for temperature.
(Default 0)
}
\par
\descrpitem{|shpar|}
\descrptext{%
Horizontal diffusion parameter for salinity.
(Default 0)
}
\par


\paragraph{Concentrations}

The next parameters deal with the transport and diffusion
of a conservative substance. The substance is dissolved in
the water and acts like a tracer.

\descrpitem{|iconz|}
\descrptext{%
Flag if the tracer computation is done.
A value different from 0 computes the transport
and diffusion of the substance. If greater than 1
|iconz| concentrations are simulated. (Default 0)
}
\par
\descrpitem{|conref|}
\descrptext{%
Reference (initial) concentration of the tracer in
any unit. (Default 0)
}
\par
\descrpitem{|contau|}
\descrptext{%
If different from 0 simulates decay of concentration. In
this case |contau| is the decay rate (e-folding time) in days.
(Default 0)
}
\par

\descrpitem{|chpar|}
\descrptext{%
Horizontal diffusion parameter for the tracer.
This value overwrites the general parameter for
horizontal diffusion |dhpar|. (Default 0)
}
\par


\paragraph{Output for scalars}

The next parameters define the output frequency of the
computed scalars (temperature, salinity, generic concentration) to file.

\descrpitem{|idtcon|\\|itmcon|}
\descrptext{%
Time step and start time for writing to file
CON (concentration) and NOS (temperature and
salinity).
}
\par

\subsection{Section {\tt \$proj}}
The parameters set in this section handle the projection from cartesian to
geographical coordinate system. If |iproj| $>$0 the projected geographical
coordinates can be used for spatially computing the Coriolis parameter
and tidal potential even if the basin has been set in a cartesian coordinate system
(|isphe| = 0) .

Please find all details here below.

\descrpitem{|iproj|}
\descrptext{%
Switch that indicates the type of projection
(default 0):
\begin{description}
\item[0] do nothing
\item[1] Gauss-Boaga (GB)
\item[2] Universal Transverse Mercator (UTM)
\item[3] Equidistant cylindrical (CPP)
\item[4] UTM non standard
\end{description}
}
\par

\descrpitem{|c\_fuse|}
\descrptext{%
Fuse for GB (1 or 2, default 0)
}
\par

\descrpitem{|c\_zone|}
\descrptext{%
Zone for UTM (1-60, default 0)
}
\par

\descrpitem{|c\_lamb|}
\descrptext{%
Central meridian for non-std UTM (default 0)
}
\par

\descrpitem{|c\_x0|}
\descrptext{%
x0 for GB and UTM (default 0)
}
\par

\descrpitem{|c\_y0|}
\descrptext{%
y0 for GB and UTM (default 0)
}
\par

\descrpitem{|c\_skal|}
\descrptext{%
Scale factor for non-std UTM (default 0.9996)
}
\par

\descrpitem{|c\_phi|}
\descrptext{%
Central parallel for CPP (default 0.9996)
}
\par

\descrpitem{|c\_lon0|}
\descrptext{%
Longitude origin for CPP (default 0)
}
\par

\descrpitem{|c\_lat0|}
\descrptext{%
Latitude origin for CPP (default 0)
}
\par

\subsection{Section {\tt \$waves}}

Parameters in this section activate the wind wave module and define
which kind of wind wave model has to be used. These parameters must
be in section |waves|.

\descrpitem{|iwave|}
\descrptext{%
Type of wind wave model and coupling procedure (default 0):
\begin{description}
    \item[0] No wind wave model called
    \item[1] The parametric wind wave model is called
    (see file subwave.f)
    \item[$>$1] The spectral wind wave model WWMIII is called
    \item[2] ... wind from SHYFEM, radiation stress formulation
    \item[3] ... wind from SHYFEM, vortex force formulation
    \item[4] ... wind from WWMIII, radiation stress formulation
    \item[5] ... wind from WWMIII, vortex force formulation
    \item[11] The spectral wind wave model WaveWatch WW3 is called
\end{description}
}
\par
When the vortex force formulation is chosen the wave-supported
surface stress is subtracted from the wind stress, in order to
avoid double counting of the wind forcing in the flow model.
Moreover, the use of the wave-dependent wind drag coefficient
can be adopted setting |itdrag| = 3.

\descrpitem{|dtwave|}
 \descrptext{%
Time step for coupling with WWMIII. Needed only for
}
\par

 \descrpitem{|iwave|} \descrptext{%
$>$1 (default 0).
}
\par
\descrpitem{|idtwav|} \descrpitem{|itmwav|} \descrptext{%
Time step and start time for writing to file wav,
the files containing output wave variables (significant
wave height, wave period, mean wave direction).
}
\par
\subsection{Section {\tt \$sedtr}}


The following parameters activate the sediment transport module
and define the sediment grain size classes to be simulated.

\descrpitem{|sedtr|}
\descrptext{%
Sediment transport module section name.
}
\par

\descrpitem{|isedi|}
\descrptext{%
Flag if the sediment computation is done:
\begin{description}
\item[0] Do nothing (default)
\item[1] Compute sediment transport
\end{description}
}
\par

\descrpitem{|idtsed|\\|itmsed|}
\descrptext{%
Time step and start time for writing to files sed e sco,
the files containing sediment variables and suspended
sediment concentration.
}
\par

\descrpitem{|sedgrs|}
\descrptext{%
Sediment grain size class vector [mm]. Values has to be
ordered from the finest to the coarser. \\
|example: sedgrs = '0.1 0.2 0.3 0.4'|
}
\par

\descrpitem{|irocks|}
\descrptext{%
Element type in which erosion-deposition is not computed
(Default -1).
}
\par

\descrpitem{|sedref|}
\descrptext{%
Initial sediment reference concentration [kg/m3]
(Default 0).
}
\par

\descrpitem{|sedhpar|}
\descrptext{%
Sediment diffusion coefficient (Default 0).
}
\par

\descrpitem{|adjtime|}
\descrptext{%
Time for sediment initialization [s]. The sediment model needs a
initialization time in which the system goes to a quasi steady
state. When t = |adjtime| the bed evolution change in the output
is reset. Keep in mind that |adjtime| has to be chosen case by
case as a function of the morphology and the parameters used for
the simulation (Default 0).
}
\par

\descrpitem{|percin|}
\descrptext{%
Initial sediment distribution (between 0 and 1) for each
grain size class. The sum of percin must be equal to 1. \\
|example: percin = 0.25 0.25 0.25 0.25| \\
If percin is not selected the model impose equal
percentage for each grainsize class (percin = 1/nrs).
In case of spatial differentiation of the sediment
distribution set a number of percin equal to the number
of grain size classes per number of area types. Element
types should be numbered consecutively starting from 0.\\
|example: percin = 0.25 0.25 0.25 0.25  \\
0.20 0.20 0.30 0.30  \\
0.45 0.15 0.15 0.15|
}
\par

\descrpitem{|tauin|}
\descrptext{%
Initial dry density or TAUCE. In function of value: \\
0-50 : critical erosion stress (Pa) \\
\textgreater 50  : dry bulk density of the surface (kg/m**3). \\
In case of spatial differentiation set a number of tauin
equal to the number of area type. Element types should be
numbered consecutively starting from 0. \\
|example: tauin = '0.9 1.4 2.5 1.1'|
}
\par

\descrpitem{|sedp|}
\descrptext{%
File containing spatially varying initial sediment distribution
for each grid node. Values are in percentage of each class and
the file should be structured with number of columns equal the
number of grain size classes and the number of row equal the
number of nodes (the order should follow the internal node
numbering).
}
\par

\descrpitem{|sedt|}
\descrptext{%
File containing spatially varying initial critical erosion
stress (Pa) or dry bulk density (kg/m3). One value for each
node (the order should follow the internal node numbering).
}
\par

\descrpitem{|sedcon|}
\descrptext{%
File containing the additional constants used in sediment
model. These parameters are usually set to the indicated
default values, but can be customized for each sediment
transport simulation. The full parameter list together with
their default value and brief description is reported in
Table \ref{tab:table_sedcon}. Most of the parameters,
especially the ones for the cohesive sediments,	have been
calibrated for the Venice Lagoon. For more information about
these parameters please refer to Neumeier et
al. \cite{urs:sedtrans05} and Ferrarin et al.
\cite{ferrarin:morpho08}.
}
\par


\paragraph{Additional parameters}
The full parameter list is reported in Table \ref{tab:table_sedcon}.
An example of the settings for the |sedcon| file is given in
\Fig\figref{turbulence}. Please note that is not necessary to
define all parameters. If not defined the default value is imposed.

\begin{figure}[ht]
\begin{verbatim}
IOPT = 3
SURFPOR = 4
DOCOMPACT = 1
\end{verbatim}
\caption{Example of the sedcon file.}
\label{fig:sedcon}
\end{figure}

\begin{table}[ht]
\caption{Additional parameter for the sediment transport model to be set
in the sedcon file.}
\begin{tabular}{lcl} \hline
Name & Default value & Description \\ \hline
CSULVA  & 159.4 & Coefficient for the solid transmitted stress by Ulva \\
TMULVA  & 1.054d-3 & Threshold of motion of Ulva [Pa] \\
TRULVA  & 0.0013 & Threshold of full resuspension of Ulva [Pa] \\
E0      & 1.95d-5 & Minimum erosion rate \\
RKERO   & 5.88 & Erosion proportionality coefficient \\
WSCLAY  & 5.0 & Primary median Ws class (in the range 1:NBCONC) \\
CDISRUPT& 0.001 & Constant for turbulent floc disruption during erosion \\
CLIM1   & 0.1 & Lower limit for flocculation [kg/m3] \\
CLIM2   & 2.0 & Limit between simple/complex flocculation [kg/m3] \\
KFLOC   & 0.001 & Constant K for flocculation equation \\
MFLOC   & 1.0 & Constant M for flocculation equation \\
RHOCLAY & 2600.0 &  Density of clay mineral \\
CTAUDEP & 1.0 & Scaling factor for TAUCD \\
PRS     & 0.0 & Resuspension probability [0-1] \\
RHOMUD  & 50.0 & Density of the freshly deposited mud \\
DPROFA  & 470.0 & Constants for density profile \\
DPROFB  & 150.0 & A : final deep density \\
DPROFC  & 0.015 & Define the shape (in conjunction with B and C) \\
DPROFD  & 0.0 & Aux parameter for density profile \\
DPROFE  & 0.0 & Aux parameter for density profile \\
CONSOA  & 1d-5 & time constant of consolidation \\
TEROA   & 6d-10 & Constant for erosion threshold from density \\
TEROB   & 3.0 & Aux parameter for erosion threshold from density \\
TEROC   & 3.47 & Aux parameter for erosion threshold from density \\
TEROD   & -1.915 & Aux parameter for erosion threshold from density \\
KCOES   & 0.15 & Fraction of mud for sediment to be cohesive \\
CDRAGRED& -0.0893 & Constant for the drag reduction formula \\
Z0COH   & 2.0D-4 & Bed roughness length for cohesive sediments \\
FCWCOH  & 2.2D-3 & Friction factor for cohesive sediments \\
LIMCOH  & 0.063 & Limit of cohesive sediment grainsize [mm] \\
SMOOTH  & 1.0 & Smoothing factor for morphodynamic \\
ANGREP  & 32.0 & Angle of repose \\
IOPT    & 5 & Sediment bedload transport formula option number \\
MORPHO  & 1.0 & Morphological acceleration factor \\
RHOSED  & 2650.0 & Sediment grain density \\
POROS   & 0.4 & Bed porosity [0-1] \\
SURFPOR & 0.6 & Bed porosity of freshly deposited sand [0-1] \\
DOCOMPACT& 0.0 & If not zero, call COMPACT routine \\ \hline
\end{tabular}
\label{tab:table_sedcon}
\end{table}

\subsection{Section {\tt \$wrt}}


Parameters for computing water renewal time.
During runtime it writes a .jas file with timeseries of total tracer
concentration in the basin and WRT computed according to different methods.
Nodal values of computed WRT are written in the .wrt file.
Frequency distributions of WRTs are written in the .frq file.

Please find all details here below.

\descrpitem{|idtwrt|}
\descrptext{%
Time step to reset concentration to c0. Use 0 if no reset
is desired. Use -1 if no renewal time computation is desired
(Default -1).
}
\par

\descrpitem{|itmin|}
\descrptext{%
Time from when to compute renewal time (-1 for start of sim)
(Default -1)
}
\par

\descrpitem{|itmax|}
\descrptext{%
Time up to when to compute renewal time (-1 for end of sim)
(Default -1).
}
\par

\descrpitem{|c0|}
\descrptext{%
Initial concentration of tracer (Default 1).
}
\par

\descrpitem{|iaout|}
\descrptext{%
Area code of elements out of the body of water (e.g. lagoon, lake, sea) (used for init and retflow).
Use -1 if no outside areas exist. (Default -1).
}
\par

\descrpitem{|percmin|}
\descrptext{%
Percentage to reach after which the computation is stopped.
Use 0 if no premature end is desired (Default 0).
}
\par

\descrpitem{|iret|}
\descrptext{%
Equal to 1 if return flow is used. If equal to 0 the
concentrations outside are explicitly set to 0 (Default 1).
}
\par

\descrpitem{|istir|}
\descrptext{%
If equal to 1 simulates completely stirred tank
(replaces at every time step conz with average conz)
(Default 0).
}
\par

\descrpitem{|iadj|}
\descrptext{%
Adjust renewal time for tail of distribution (Default 1).
}
\par

\descrpitem{|ilog|}
\descrptext{%
Use logarithmic regression to compute renewal time (Default 0).
}
\par

\descrpitem{|ctop|}
\descrptext{%
Maximum to be used for frequency curve (Default 0).
}
\par

\descrpitem{|ccut|}
\descrptext{%
Cut renewal time at this level (for res time computation)
(Default 0).
}
\par

\descrpitem{|wrtrst|}
\descrptext{%
If reset times are not regularly distributed (e.g., 1 month)
it is possible to give the exact times when a reset should
take place. |wrtrst| is a file name where these reset times
are specified, one for each line. For every line two integers
indicating date and time for the reset must be specified.
If only one value is given, time is taken to be 0. The format
of date is "YYYYMMDD" and for time "hhmmss". If the file
wrtrst is given |idtwrt| should be 0.
}
\par

\subsection{Section {\tt \$lagrg}}


This section describes the use of the Lagrangian Particle Module.
The lagrangian particles can be released inside a specified area with a
regular distribution. The area is defined in the file |lagra|.
The amount of particles released and the
time step is specified by |nbdy| and |idtrl|.

The lagrangian module runs between the times |itlanf| and |itlend|. If one
or both are missing, the simulation extremes are substituted. Inside
the lagrangian simulation window the release of particles is controlled
by the parameters |idtl|, |itranf| and |itrend|. |itranf| gives the time
of the first release, |itrend| the time for the last release. If not
given they are set equal to the extremes of the lagrangian simulation.
|idtl| gives the time step of release.

Particles are released inside the given areas (filename |lagra|). If
this file is not specified they are released over the whole domain. It is also possible to
 release particles over open boundaries. However,
this is still experimental. Please see the file |lagrange_main.f|
for more details.

The output frequency of the results can be contolled by
|idtlgr| and |itmlgr|.

Please find all details here below.

\descrpitem{|ilagr|}
\descrptext{%
Switch that indicates if the lagrangian module
should be run (default 0):
\begin{description}
\item[0] do nothing
\item[1] surface lagrangian
\item[2] 2d lagrangian (not implemented)
\item[3] 3d lagrangian
\end{description}
}
\par

\descrpitem{|nbdymax|}
\descrptext{%
Maximum numbers of particles that can be in the domain.
This should be the maximum number of particles
that can be created and inserted. Use 0 to not limit
the number of particles (at your own risk). This
parameter must be set and has no default.
}
\par

\descrpitem{|nbdy|}
\descrptext{%
Total number of particles to be released in the domain each
time a release of particles takes place.
(Default 0)
}
\par

\descrpitem{|rwhpar|}
\descrptext{%
A horizontal diffusion can be defined for the lagrangian model.
Its value can be specified in |rwhpar| and the units are
[m**2/s]. (Default 0)
}
\par

\descrpitem{|itlanf\\itlend|}
\descrptext{%
The start and end time for the lagrangian module.
If not given, the module runs for the whole simulation.
}
\par

\descrpitem{|itmlgr\\idtlgr|}
\descrptext{%
Initial time and time step for the output to file
of the particles. if |idtlgr| is 0,
no output is written. (Default 0)
}
\par

\descrpitem{|idtl|}
\descrptext{%
The time step used for the release of particles. If this
is 0 particles are released only once at the beginning
of the lagrangian simulation. No particles are released
for a value of less than 0. (Default 0)
}
\par

\descrpitem{|itranf\\itrend|}
\descrptext{%
Initial and final time for the release of particles.
If not specified the particles are released over the
whole lagrangian simulation period.
}
\par

\descrpitem{|ipvert|}
\descrptext{%
Set the vertical distribution of particles:
\begin{description}
\item[0] releases one particle only in surface layer
\item[$>$0] releases n particles regularly
\item[$<$0] releases n particles randomly
\end{description}
}
\par

|linbot| Set the bottom layer for vertical releases (Default -1, bottom layer)
|lintop| Set the top layer for vertical releases (Default 1, surface layer)

\descrpitem{|lagra|}
\descrptext{%
File name that contains closed lines of the area where
the particles have to be released. If not given, the particles
are released over the whole domain.
}
\par

\subsection{Section {\tt \$name}}

In section |name| the names of input files can be
given. All directories default to the current directory,
whereas all file names are empty, i.e., no input files are
given.
Strings in section |name| enable the specification of files
c that account for initial conditions or forcing.

\descrpitem{|zinit|}
\descrptext{%
Name of file containing initial conditions for water level.
}
\par

\descrpitem{|uvinit|}
\descrptext{%
Name of file containing initial conditions for velocity
}
\par

\descrpitem{|wind|}
\descrptext{%
File with wind data. The file may be either
formatted or unformatted. For the format of the unformatted
file please see the section where the WIN
file is discussed.
The format of formatted ASCII file
is in standard time-series format, with the
first column containing the time in seconds and
the next two columns containing the wind data.
The meaning of the two values depend on the
value of the parameter |iwtype| in the |para| section.
}
\par
\descrpitem{|qflux|}
\descrptext{File with heat flux data. This file must be in
a special format to account for the various parameters
that are needed by the heat flux module to run. Please
refer to the information on the file \descrpitem{|qflux|}.}

\descrpitem{|rain|}
\descrptext{File with rain data. This file is a standard time series
with the time in seconds and the rain values
in mm/day. The values may include also evaporation. Therefore,
also negative values (for evaporation) are permitted.}

\descrpitem{|ice|}
\descrptext{File with ice cover. The values range from 0 (no ice cover)
to 1 (complete ice cover).}

\descrpitem{|surfvel|}
\descrptext{File with surface velocities from observation. These
data can be used for assimilation into the model.}

\descrpitem{|restrt|}
\descrptext{Name of the file if a restart is to be performed. The
file has to be produced by a previous run
with the parameter |idtrst| different
from 0. The data record to be used in the file for the
restart must be given by time |itrst|.}

\descrpitem{|gotmpa|}
\descrptext{Name of file containing the parameters for the
GOTM turbulence model (iturb = 1).}

\descrpitem{|tempin|}
\descrptext{Name of file containing initial conditions for temperature}

\descrpitem{|saltin|}
\descrptext{Name of file containing initial conditions for salinity}

\descrpitem{|conzin|}
\descrptext{Name of file containing initial conditions for concentration}

\descrpitem{|tempobs|}
\descrptext{Name of file containing observations for temperature}

\descrpitem{|saltobs|}
\descrptext{Name of file containing observations for salinity}

\descrpitem{|temptau|}
\descrptext{Name of file containing the time scale for nudging
of temperature}

\descrpitem{|salttau|}
\descrptext{Name of file containing the time scale for nudging
of salinity}

\descrpitem{|bfmini|}
\descrptext{Name of file containing initial conditions for bfm}

\descrpitem{|offlin|}
\descrptext{Name of the file if a offline is to be performed. The
file has to be produced by a previous run
with the parameter |idtoff| greater than 0.}

\subsection{Section {\tt \$bound}}

These parameters determine the open boundary nodes and the type
of the boundary: level or flux boundary. At the first type of boundary the water levels
are imposed, at the second the fluxes are prescribed.

There may be multiple sections |bound| in one parameter input file,
describing all open boundary conditions necessary. Every section
must therefore be supplied with a boundary number. The numbering
of the open boundaries must
be increasing. The number of the boundary must be specified
directly after the keyword |bound|, such as |bound1| or |bound 1|.

\descrpitem{|kbound|}
\descrptext{%
Array containing the node numbers that are part of the
open boundary. The node numbers must form one contiguous
line with the domain (elements) to the left. This
corresponds to an anti-clockwise sense. The type of
boundary depends on the	value of |ibtyp|. In case this value
is 1 or 2 at least two nodes must be given.
}
\par

\descrpitem{|ibtyp|}
\descrptext{%
Type of open boundary.
\begin{description}
\item[0] No boundary values specified
\item[1] Level boundary. At this open boundary
the water level is imposed and the prescribed
values are interpreted as water levels in meters.
If no value for |ibtyp| is specified this
is the default.
\item[2] Flux boundary. Here the discharge in \dischargeunit
has to be prescribed.
\item[3] Internal flux boundary. As with |ibtyp = 2| a
discharge has to be imposed, but the node where
discharge is imposed can be an internal node
and need not be on the outer boundary of
the domain. For every node in |kbound| the
volume rate specified will be added to the
existing water volume. This behavior is different
from the |ibtyp = 2| where the whole boundary
received the discharge specified.
\item[4] Momentum input. The node or nodes may be internal.
This feature can be used to describe local
acceleration of the water column.
The unit is force / density [\maccelunit].
In other words it is the rate of volume
[\dischargeunit] times the velocity [m/s]
to which the water is accelerated.
\end{description}
}
\par
\descrpitem{|iqual|}
\descrptext{%
If the boundary conditions for this open boundary
are equal to the ones of boundary |i|, then
setting |iqual = i| copies all the values of
boundary |i| to the actual boundary. Note that the
value of |iqual| must be smaller than the number
of the actual boundary, i.e., boundary |i| must have
been defined before. (This feature is temporarily
not working; please do not use.)
}
\par

The next parameters give a possibility to specify the file name
of the various input files that are to be read by the model.
Values for the boundary condition can be given at any time step.
The model interpolates in between given time steps if needed. The
grade of interpolation can be given by |intpol|.

All files are in ASCII and share a common format.
The file must contain two columns, the first giving the
time of simulation in seconds that refers to the value
given in the second column. The value in the second
column must be in the unit of the variable that is given.
The time values must be in increasing order.
There must be values for the whole simulation,
i.e., the time value of the first line must be smaller
or equal than the start of the simulation, and the time
value of the last line must be greater or equal than the
end of the simulation.

\descrpitem{|boundn|}
\descrptext{%
File name that contains values for the boundary condition.
The value of the variable given in the second column
must be in the unit determined by |ibtyp|, i.e.,
in meters for a level boundary, in \dischargeunit for
a flux boundary and in \maccelunit for a momentum
input.
}
\par
\descrpitem{|zfact|}
\descrptext{%
Factor with which the values from |boundn|
are multiplied to form the final value of the
boundary condition. E.g., this value can be used to
set up a quick sensitivity run by multiplying
all discharges by a factor without generating
a new file. (Default 1)
}
\par

\descrpitem{|levmin\\levmax|}
\descrptext{%
A point discharge normally distributes its discharge
over the whole water column. If it is important that in
a 3D simulation the water mass discharge is concentrated
only in some levels, the parameters |levmin| and |levmax|
can be used. They indicate the lowest and deepest level over
which the discharge is distributed. Default values are 0, which
indicate that the discharge is distributed over the
whole water column. Setting only |levmax| distributes from
the surface to this level, and setting only |levmin|
distributes from the bottom to this level.
}
\par

\descrpitem{|conzn\\tempn\\saltn|}
\descrptext{%
File names that contain values for the respective
boundary condition, i.e., for concentration,
temperature and salinity. The format is the same
as for file |boundn|. The unit of the values
given in the second column must the ones of the
variable, i.e., arbitrary unit for concentration,
centigrade for temperature and psu (per mille)
for salinity.
}
\par

\descrpitem{|vel3dn|}
\descrptext{%
File name that contains current velocity values for the
boundary condition.  The format is the same as for file
|tempn| but it has two variables:
current velocity in x and current velocity in y.
Velocity can be nudged or imposed depending on the value
of |tnudge| (mandatory). The unit is [m/s].
}
\par

\descrpitem{|tnudge|}
\descrptext{%
Relaxation time for nudging of boundary velocity.
For |tnudge| = 0 velocities are imposed, for
|tnudge| $>$ 0 velocities are nudged. The
default is -1 which means do nothing. Unit is [s].
(Default -1)
}
\par

The next variables specify the name of the boundary value file
for different modules. Please refer to the documentation of the
single modules for the units of the variables.

\descrpitem{|bio2dn|}
\descrptext{%
File name that contains values for the ecological
module (EUTRO-WASP).
}
\par
\descrpitem{|sed2dn|}
\descrptext{%
File name that contains values for the sediment
transport module.
The unit of the values given
in the second and following columns (equal to the
number of defined grainsize in parameter |sedgrs|).
}
\par

\descrpitem{|mud2dn|}
\descrptext{%
File name that contains values for the fluid mud
module.
}
\par
\descrpitem{|lam2dn|}
\descrptext{%
File name that contains values for the fluid mud
module (boundary condition for the structural parameter,
to be implemented).
}
\par
\descrpitem{|dmf2dn|}
\descrptext{%
File name that contains values for the fluid mud
module (boundary conditions for the advection of flocsizes,
to be implemented).
}
\par
\descrpitem{|tox3dn|}
\descrptext{%
File name that contains values for the toxicological
module.
}
\par

\descrpitem{|bfmbcn|}
\descrptext{%
File name that contains values for the bfm module.
}
\par

\descrpitem{|mercn|}
\descrptext{%
File name that contains values for the mercury module.
}
\par

\descrpitem{|intpol|}
\descrptext{%
Order of interpolation for the boundary values read
in files. Use 1 for stepwise (no) interpolation,
2 for linear and 4 for cubic interpolation.
The default is linear interpolation, except for
water level boundaries (|ibtyp=1|) where cubic
interpolation is used.
}
\par

The next parameters can be used to impose a sinusoidal water level
(tide) or flux at the open boundary. These values are used if no
boundary file |boundn| has been given. The values must be in the unit
of the intended variable determined by |ibtyp|.

\descrpitem{|ampli|}
\descrptext{%
Amplitude of the sinus function imposed. (Default 0)
}
\par
\descrpitem{|period|}
\descrptext{%
Period of the sinus function. (Default 43200, 12 hours)
}
\par
\descrpitem{|phase|}
\descrptext{%
Phase shift of the sinus function imposed. A positive value
of one quarter of the period reproduces a cosine
function. (Default 0)
}
\par
\descrpitem{|zref|}
\descrptext{%
Reference level of the sinus function imposed. If only
|zref| is specified (|ampli = 0|) a constant value
of |zref| is imposed on the open boundary.
}
\par

With the next parameters a constant value can be imposed for the
variables of concentration, temperature and salinity. In this case
no file with boundary values has to be supplied. The default for all
values is 0, i.e., if no file with boundary values is supplied and
no constant is set the value of 0 is imposed on the open boundary.
A special value of -999 is also allowed. In this case the value
imposed is the ambient value of the parameter close to the boundary.

\descrpitem{|conz\\temp\\salt|}
\descrptext{%
Constant boundary values for concentration,
temperature and salinity respectively. If these
values are set no boundary file has to be supplied.
(Default 0)
}
\par

The next two values are used for constant momentum input.
This feature can be used to describe local acceleration of the
water column. The values give the input of momentum
in x and y direction. The unit is force / density (\maccelunit).
In other words it is the rate of volume (\dischargeunit) times
the velocity (m/s) to which the water is accelerated.

These values are used if
boundary condition |ibtyp = 4| has been chosen and
no boundary input file has been given.
If the momentum input is varying then it may be specified with
the file |boundn|. In this case the file |boundn| must contain
three columns, the first for the time, and the other two for
the momentum input in $x,y$ direction.

Please note that this feature is temporarily not available.

\descrpitem{|umom\\vmom|}
\descrptext{%
Constant values for momentum input. (Default 0)
}
\par

The next two values can be used
to achieve the tilting of the open boundary if only one water level value
is given. If only |ktilt| is given then the boundary values
are tilted to be in equilibrium with the Coriolis force. This may avoid
artificial currents along the boundary. |ktilt| must be a boundary node
on the boundary.

If |ztilt| is given the tilting of the boundary is explicitly set
to this value. The tilting of the first node of the boundary is set
to $-|ztilt|$
and the last one to $+|ztilt|$. The total amount of tilting is
therefore is $2 \cdot |ztilt|$. If |ktilt| is not specified
then a linear interpolation between the first and the last boundary
nodes will be carried out. If also |ktilt| is specified then
the boundary values are arranged that the water levels are
tilted around |ktilt|, e.g., $-|ztilt|$ at the first boundary node,
0 at |ktilt|, and $+|ztilt|$ at the last boundary node.

\descrpitem{|ktilt|}
\descrptext{%
Node of boundary around which tilting should
take place. (Default 0, i.e., no tilting)
}
\par
\descrpitem{|ztilt|}
\descrptext{%
Explicit value for tilting (unit meters).
(Default 0)
}
\par

Other parameters:

\descrpitem{|igrad0|}
\descrptext{%
If different from 0, a zero gradient boundary
condition will be implemented. This is already the
case for scalars under outflowing conditions. However,
with |igrad0| different from 0 this conditions
will be used also for inflow conditions. (Default 0)
}
\par

\descrpitem{|tramp|}
\descrptext{%
Use this value to start smoothly a discharge
boundary condition. If set, it indicates the
time (seconds) that will be used to increase
a discharge from 0 to the desired value (Default 0)
}
\par

\descrpitem{|levflx|}
\descrptext{%
If discharge is depending on the water level
(e.g., lake outflow) then this parameter indicates to
use one of the possible outflow curves. Please
note that the flow dependence on the water level
must be programmed in the routine $|level\_flux()|$.
(Default 0)
}
\par

\descrpitem{|nad|}
\descrptext{%
On the open boundaries it is sometimes convenient
to not compute the non-linear terms in the momentum
equation because instabilities may occur. Setting
the parameter |nad| to a value different from 0
indicates that in the first |nad| nodes from the
boundary the non linear terms are switched off.
(Default 0)
}
\par

\descrpitem{|lgrpps|}
\descrptext{%
Indicates the number of particles released at
the boundary for the lagrangian module. If positive,
it is the number of particles per second released
along the boundary. If negative, its absolute
value indicates the particles per volume flux
(unit \dischargeunit) released along the boundary.
(Default 0)
}
\par

\subsection{Section {\tt \$wind}}


In this section the wind data can be given directly without
the creation of an external file. Note, however, that
a wind file specified in the |name| section takes precedence
over this section. E.g., if both a section |wind| and a
wind file in |name| is given, the wind data from the file is used.

The format of the wind data in this section is the same as the
format in the ASCII wind file, i.e., three columns, with
the first specifying the time in seconds and the other two columns
giving the wind data. The interpretation of the wind data
depends on the value of |iwtype|. For more information please
see the description of |iwtype| in section |para|.

\subsection{Section {\tt \$extra}}

In this section, the node numbers of so called ``extra'' points are given.
These are points where the value of simulated variables (water level,
velocities, temperature, salinity, tracer, etc.) are written to create
a time series that can be elaborated later. The output for these ``extra''
points consumes little memory and can be therefore written with a
much higher frequency (typically the same as the integration time step)
than the complete hydrodynamic output. The output is written
to file EXT.

The format of the section is the following:
\begin{verbatim}
$extra
        node1   'string1'
        node2   'string2'
        etc..
$end
\end{verbatim}
where node is the node number and string is a description of the node.
If no description strings are needed the nodes can also be specified
by just giving their values:
\begin{verbatim}
$extra
        node1  node2  node3
        node4  etc..
$end
\end{verbatim}
This format is however deprecated.


\subsection{Section {\tt \$flux}}

In this section, transects are specified through which the discharge
of water is computed by the program and written to file FLX.
The transects are defined by their nodes through which they run.
All nodes in one transect must be adjacent, i.e., they must form a
continuous line in the FEM network.

The nodes of the transects are specified in free format and are
ended with the description of the section.
An example is given here:
\begin{verbatim}
$flux
        1001 1002 1004 'section 1'
        35 37 46 'special section'
        407
        301 'section given on two lines'
$end
\end{verbatim}
The example shows the definition of 3 transects. As can be seen, the
nodes of the transects can be given on one line alone (first transect),
or on more than one lines (transect 3).
There is also an old format that separates one section from the other by
inserting the value 0. However, this format is deprecated.

