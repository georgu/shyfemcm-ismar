SHYFEM provides a number of routines to pre-process grd files which support the
-h option. Below they are described in their full functionalities.

\subsection{exgrd}
|exgrd| is the tool to extract items from GRD File.
\begin{verbatim}
Usage : exgrd [-h] [-options] grd-file(s)
Options :
  -i   show only info on file          -h   show this help screen
  -n   extract nodes                   -N   delete nodes
  -e   extract elements                -E   delete elements
  -l   extract lines                   -L   delete lines
  -s   extract unused nodes            -S   delete unused nodes
  -t#  select from type                -T#  select up to type
  -d#  select from depth               -D#  select up to depth
  -v#  select from vertices            -V#  select up to vertices
  -r#  select from number range        -R#  select up to number range
  -u   unify nodes                     -o#  tollerance for unify (def=0)
  -C   compress numbers                -a   make items anti-clockwise
  -x   delete degenerate elements
\end{verbatim}

\subsection{shybas}
|shybas| is the routine for elaborating grid files (GRD and BAS). It also returns
information on grid file.

\begin{verbatim}
 Usage: shybas [-h|-help] [-options] bas-file
  returns info on or elaborates a bas file
  options:
   -h|-help        this help screen
   -v|-version     version of routine
   -verb           be more verbose
   -quiet          do not be verbose
   -nomin          do not compute min distance
   -area           area/vol for each area code
  output options:
   -grd            writes grd file
   -xyz            writes xyz file
   -depth          writes depth values
   -reg dxy        writes regular depth values
   -unique         writes grd file with unique depths on nodes
   -delem          writes grd file with constant depths on elements
   -hsigma val     creates hybrid depth level
   -npart          writes grd file with nodal partition to be visualized
   -part grd-file  uses lines contained in grd-file to create partition
   -gr3            writes grid in gr3 format (for WWMIII model
   -msh            writes grid in msh (gmsh v. 2) format (for WW3 model
  what to do:
   -inter          interactively shows nodes and elems
   -quality        shows quality of file
   -resol          writes resolution of file
   -freq           computes frequency curves
   -check          runs extra check on file
   -compare        compares depth of 2 basins
   -invert_depth   inverts depth values
   -box            creates index for box model
   -custom         run custom routine defined by user
  bathymetry interpolation:
   -bfile bathy    bathymetry file for interpolation
   -all            interpolate in all elements
   -btype type     interpolate only on elems type (Default -1)
   -node           interpolate to nodes
   -bmode mode     mode of interpolation
   -usfact fact    factor for std (Default 1)
   -uxfact fact    factor for max radius (Default 3)
  limiting and smoothing bathymetry:
   -hmin val       minimum depth
   -hmax val       maximum depth
   -asmooth alpha  alpha for smoothing
   -iter n         iterations for smoothing

   interpolation mode: (default=1)
     1 exponential
     2 uniform on squares
     3 exponential with autocorrelation
\end{verbatim}

\subsection{shypre}
The pre-processing routine |shypre| is used to generate the bas unformatted
file from the grd formatted file

\begin{verbatim}
 Usage: shypre [-h|-help] [-options] grd-file
  pre-processes grd file and create bas file
  options:
   -h|-help         this help screen
   -v|-version      version of routine
  general options
   -info            only give info on grd file
   -quiet           be quiet in execution
   -silent          do not write anything
  optimization options
   -noopti          do not optimize bandwidth
   -manual          manual optimization
  options for partitioning
   -partition file  use file containing partitioning
   -nepart          use node and elem type in file for partitioning
\end{verbatim}

\subsection{shyparts}
This program performs the partition of a grid using the METIS library.

\begin{verbatim}
 Usage: shyparts [-h|-help] [-options] grd-file
  partitioning of grd file with METIS
  options:
   -h|-help      this help screen
   -v|-version   version of routine
  options for partitioning
   -nparts val   number of partitions
\end{verbatim}
