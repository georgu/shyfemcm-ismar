
%------------------------------------------------------------------------
%
%    Copyright (C) 1985-2020  Georg Umgiesser
%
%    This file is part of SHYFEM.
%
%    SHYFEM is free software: you can redistribute it and/or modify
%    it under the terms of the GNU General Public License as published by
%    the Free Software Foundation, either version 3 of the License, or
%    (at your option) any later version.
%
%    SHYFEM is distributed in the hope that it will be useful,
%    but WITHOUT ANY WARRANTY; without even the implied warranty of
%    MERCHANTABILITY or FITNESS FOR A PARTICULAR PURPOSE. See the
%    GNU General Public License for more details.
%
%    You should have received a copy of the GNU General Public License
%    along with SHYFEM. Please see the file COPYING in the main directory.
%    If not, see <http://www.gnu.org/licenses/>.
%
%    Contributions to this file can be found below in the revision log.
%
%------------------------------------------------------------------------

Once the |str| file (e.g., |param.str|) is made, the following line
starts the simulation
\begin{verbatim}
	shyfem param.str
\end{verbatim}

\subsection{Structure}

The input parameter file is the file that guides program performance. It
contains all the necessary informations for the main routine to execute
the model. Nearly all parameters that can be given have a default value
which is used when the parameter is not listed in the file. Only some
time parameters are compulsory and must be present in the file.

The format of the file looks very like a namelist format, but is
not dependent on the compiler used. Values of parameters are given
in the form :  
|name = value|  or  |name = 'text'|.  If |name|
is an array the following format is used : 
\begin{verbatim}
          name = value1 , value2, ... valueN
\end{verbatim}
The list can continue on the following lines. Blanks before and after
the equal sign are ignored. More then one parameter can be present
on one line. As separator blank, tab and comma can be used.

Parameters, arrays and data must be given in between certain sections.
A section starts with the character {\tt \$} followed by a keyword and
ends with {\tt \$end}. The {\tt \$keyword} and {\tt \$end} must not
contain any blank characters and must be the first non blank characters
in the line. Other characters following the keyword on the same line
separated by a valid separator are ignored.

Several sections of data may be present in the input parameter file.
Further ahead all sections are presented and the possible
parameters that can be specified are explained. The sequence in
which the sections appear is of no importance. However, the first 
section must always be section |\$title|, the section that
determines the name of simulation and the basin file to use and
gives a one line description of the simulation.

Lines outside of the sections are ignored. This gives
the possibility to comment the parameter input file.

Figure \ref{fig:example} shows an example of a typical input
parameter file and the use of the sections and definition of
parameters.

\importstr{example}
{Example of a parameter input file ({\tt STR} file)}

