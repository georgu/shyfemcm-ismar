
SHYFEM provides also routines to inquire and elaborate shy files. Below the list with the description of their full
functionalities.
\subsection{shyelab}
|shyelab |is the routine for elaborating SHY files. It also returns
information on a SHY file.

\begin{verbatim}
  Usage: shyelab [-h|-help] [-options] shy-file
  returns info on or elaborates a shyfem file
  options:
   -h|-help              this help screen
   -v|-version           version of routine
  general options
   -info                 only give info on header
   -verbose              be more verbose, write time records
   -quiet                do not write header information
   -silent               do not write anything
   -write                write min/max of records
  time options
   -tmin time            only process starting from time
   -tmax time            only process up to time
   -inclusive            output includes whole time period given
   -changetime difftime  add difftime to time record (difftime [s])
      time is either YYYY-MM-DD[::hh[:mm[:ss]]]
      or integer for relative time
   -rmin rec             only process starting from record rec
   -rmax rec             only process up to record rec
   -rfreq freq           only process every freq record
      rec in rmax can be negative
      this indicates rec records from the back
  output options
   -out                  writes new file
   -outformat form       output format
      possible output formats are: shy|gis|fem|nc|off (Default native)
      not all formats are available for all file types
   -catmode cmode        concatenation mode for handeling more files
      possible values for cmode are: -1,0,+1 (Default 0)
      -1: all of first file, then remaining of second
       0: simply concatenate files
      +1: first file until start of second, then all of second
   -proj projection      projection of coordinates
      projection is string consisting of mode,proj,params
      mode: +1: cart to geo,  -1: geo to cart
      proj: 1:GB, 2:UTM, 3:CPP
  extract options
   -split                split file for variables
   -splitall             splits file (EXT and FLX) for extended data
   -check period         checks data over period
    period can be all,year,month,week,day,none
   -checkdt              check for change of time step
   -checkrain            check for yearly rain (if file contains rain)
   -node nlist           extract vars of nodes in list
      nlist is a comma separated list of nodes to be extracted
   -nodes nfile          extract vars at nodes given in file nfile
      nfile is a file with nodes to be extracted
   -extract recs         extract records specified in recs
      recs is either a comma separated list like r1,r2,r3
      or in the format istart..ifreq..iend (..iend may be missing)
   -coord coord          extract coordinate
      coord is x,y of point to extract
  time series options
   -convert              convert time column to ISO string
   -date0 string         reference date for conversion of time column
   -facts fstring        apply factors to data in data-file
   -offset ostring       apply factors to data in data-file
      fstring and ostring is comma separated factors, empty for no change
  specific FEM file options
   -condense             condense file data into one node
   -chform               change output format form/unform of FEM file
   -grd                  write GRD file from data in FEM file
   -grdcoord             write regular coordinates in GRD format
   -nodei node           extract internal node
      node is internal numbering in fem file or ix,iy of regular grid
   -newstring sstring    substitute string description in fem-file
      sstring is comma separated strings, empty for no change
  regular output file options
   -reg rstring          regular interpolation
   -resample bounds      resample regular grid
   -regexpand iexp       expand regular grid
      rstring is: dx[,dy[,x0,y0,x1,y1]]
      if only dx is given -> dy=dx
      if only dx,dy are given -> bounds computed
      bounds is: x0,y0,x1,y1
      iexp>0 expands iexp cells, =0 whole grid
      resample should be used with regexpand
  transformation of SHY file options
   -averbas              average over basin
   -aver                 average over records
   -averdir              average for directions
   -sum                  sum over records
   -min                  minimum of records
   -max                  maximum of records
   -std                  standard deviation of records
   -rms                  root mean square of records
   -sumvar               sum over variables
   -threshold t          compute records over threshold t
   -fact fact            multiply values by fact
   -freq n               frequency for aver/sum/min/max/std/rms
   -2d                   average vertically to 2d field
   -vorticity            compute vorticity for hydro file
  difference of SHY file options
   -diff                 check if 2 files are different
   -diffeps deps         files differ by more than deps (default 0)
      this option needs two files and exits at difference
      with -out writes difference to out file
  specific LGR file options
   -lgmean               extract mean position and age in function
       of particle type. It  write files lagrange_mean_traj.*
   -lgdens               compute distribution of particle density and age
       either on nodes or on regular grid (using -reg option)
   -lg2d                 sum particles vertically when computing the
       particle density/age
   -lgtype               compute density per type
   -nlgtype max-type     max number of types to compute
  All options for all file types are shown
  Not all options are available for all files
  To show options for a specific file use: shyelab -h file
\end{verbatim}

