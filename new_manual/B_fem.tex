The FEM file uses an optimized way of specifying variables that can be used
in SHYFEM for setting boundary and initial conditions. For 3D files, only
variables that are exisiting are stored. This makes a big difference
when using zeta coordinates, because in the coastal zone many deep
layers are not existing.

The FEM file is headerless. What that means is that the whole file
has no specific header, whereas the single records have headers. This
feature makes it easy to combine different fem files together. If there
are two files of the same variable (say temperature) of different years
(|temp2018.fem|, |temp2019.fem|), combining them is as easy as

|cat temp2018.fem temp2019.fem > combined.fem|. 

The only thing that has
to be taken care of is that the time records could not be unique (there
might be two time records with the same time step), but this
can be handled in a second step with the program femelab:

|femelab -checkdt -out combined.fem|

which will write a new file |out.fem| with unique time stamps.

All routines to write FEM files can be found in |fem3d/subfemfile.f|.
In the same file also documentation can be found on the file format.
In the file |femutil/femfiles/profile.f| examples of how to write
FEM files can be found.

Each FEM file consists of time records:

\begin{verbatim}
!       time record 1
!       time record 2
!       time record ...
\end{verbatim}

The format of each time record is as follows:

\begin{verbatim}
!       header record
!       data record for variable 1
!       data record for variable 2
!       data record for variable ...
!       data record for variable nvar
\end{verbatim}

The format of the header is

\begin{verbatim}
!       dtime,nvers,idfem,np,lmax,nvar,ntype
!       date,time                          for ntype == 1
!       (hlv(l),l=1,lmax)                  only if( lmax > 1 )
!       regpar                             for ntype == 10
\end{verbatim}

For the explaination of the meaning of the various parameters please
see the legend below. |nvers| is a version number of the FEM files. Presently,
the version is 3.

The format of the data record depends on |lmax| which is 1 for 2D data
fields and greater than 1 for 3D fields:

\begin{verbatim}
!       if( lmax == 1 )
!               string
!               np,lmax                    only for nvers > 2
!               (data(1,k),k=1,np)
!       if( lmax > 1 )
!               string
!               np,lmax                    only for nvers > 2
!               do k=1,np
!                 lm,hd(k),(data(l,k),l=1,lm)
!               end do
\end{verbatim}

The meaning of the various parameters can be found here in the legend.
All variables and parameters are integer except where otherwise indicated.

\begin{verbatim}
! dtime         time stamp (double precision, seconds)
! nvers         version of file format
! idfem         id to identify fem file (must be 957839)
! np            number of horizontal points given
! lmax          maximum number of layers given (1 for 2D)
! nvar          number of variables in time record
! ntype         type of data, defines extra data to follow
! date          reference date (integer, YYYYMMDD)
! time          reference time (integer, hhmmss)
! hlv           layer depths (the bottom of each layer)
! string        string with description of data (character*80)
! ilhkv(k)      total number of levels of node k (1 for 2D)
! hd(k)         total depth in node k (real, -999 if unknown)
! data(l,k)     data for variable at level l and node k (real)
! lm            total number of vertical data for point k
! k,l           index for horizontal/vertical dimension
! regpar        regular grid info: nx,ny,x0,y0,dx,dy,flag
! nx,ny         size of regular grid
! x0,y0         origin of regular grid (real)
! dx,dy         space increment of regular grid (real)
! flag          flag for invalid data of regular grid (real)
\end{verbatim}

The meaning of |ntype| is given here:

\begin{verbatim}
! 0             no other lines in header
! 1             give date/time of reference on extra line
! 10            regular grid, info on extra line (regpar)
\end{verbatim}

and combinations are possible. So |ntype=11| means date/time line 
and regular grid.

One clarification on the time specification: the variable |dtime| indicates the
seconds relative to the reference date |date,time|.
Therefore, you can either use |dtime| as time
in seconds referred to the reference date specified with |date,time|. Or you
can set |dtime=0| and specify the date and time of the record in the
|date,time| line. It is also possible to mix both approaches. 

If no reference date is specified, |dtime| refers to the reference
date in the simulation. This approach is not recommended.
