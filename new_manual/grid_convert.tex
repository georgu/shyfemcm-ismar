
%------------------------------------------------------------------------
%
%    Copyright (C) 1985-2020  Georg Umgiesser
%
%    This file is part of SHYFEM.
%
%    SHYFEM is free software: you can redistribute it and/or modify
%    it under the terms of the GNU General Public License as published by
%    the Free Software Foundation, either version 3 of the License, or
%    (at your option) any later version.
%
%    SHYFEM is distributed in the hope that it will be useful,
%    but WITHOUT ANY WARRANTY; without even the implied warranty of
%    MERCHANTABILITY or FITNESS FOR A PARTICULAR PURPOSE. See the
%    GNU General Public License for more details.
%
%    You should have received a copy of the GNU General Public License
%    along with SHYFEM. Please see the file COPYING in the main directory.
%    If not, see <http://www.gnu.org/licenses/>.
%
%    Contributions to this file can be found below in the revision log.
%
%------------------------------------------------------------------------

In order to create the computational grid you need data of the coast line
and of the bathymetry.
First of all you must create a coast line in |grd| format. You can do
this with your own tools, but you could find useful the script
|coast.pl|, which converts a simple coast file (x,y) into a |grd| file.
To use it run:

\begin{code}
    coast.pl mpcoast.dat > coast.grd
\end{code}

After this step you can check your coast line with the |grid| program:

\begin{code}
    grid coast.grd
\end{code}

Likewise, if you have a bathymetry in a simple (x,y,z) format, you can
convert it with:

\begin{code}
    bathy.pl mpbathy.dat > depth.grd
\end{code}

You can check the file |depth.grd| with |grid| as well. However, this file 
will be used only after the creation of the mesh.

Please note that UTM coordinates are normally huge numbers, there
might be an accuracy problem when you try to create the grid. If this
happens, you should first shift your UTM coordinates so that the origin
of your new coordinate system coincides with the central point of your
grid. This translation can be done using the program |grd_transl.pl|.

Other transformation routines are:

\begin{itemize}

\item |dxf2grd.pl|  Transforms a grid from |dxf| (Autocad) to |grd|
format. This is still experimental.

\item |kml2grd.pl|  Transforms a grid from the Google Earth format |kml|
to |grd| format.

\item |xyz2grd.pl|  Transforms a simple list of nodes to |grd|
format. Every line contains 3 values $(x,y,z)$ or two values $(x,y)$,
when the information on depth is missing.

\end{itemize}

Please note that for SHYFEM depth values have to be positive 
for all the elements whose depth is below the chosen reference mean-sea-level.
If some very shallow grid elements have depths that are above the reference sea-level 
of your bathimetry, then only those elements can have negative depths.
If your files have depth values as negative numbers, you will have to invert
them. You can use the command

\begin{code}
    grd_modify.pl -depth_invert grd-file
\end{code}

to achieve this task.

