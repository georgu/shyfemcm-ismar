The SHY file stores the output of the SHYFEM model. It uses an optimized
way of storing variables to save space.  For 3D files, only variables
that are exisiting are stored. This makes a big difference when using
zeta coordinates, because in the coastal zone many deep layers are
not existing.

The SHY file consists of a file header and time records of various
variables.
All routines needed to write SHY files can be found in |fem3d/subshy.f|.
In the same file also documentation can be found on the file format.

Each SHY file consists of the following records:

\begin{verbatim}
!       header record
!       time record 1
!       time record 2
!       time record ...
\end{verbatim}

The time record is structured as follows:

\begin{verbatim}
!       data record for variable 1
!       data record for variable 2
!       data record for variable ...
!       data record for variable nvar
\end{verbatim}

The format of the header is:

\begin{verbatim}
!        write(iunit,err=99) idshy,nvers
!        write(iunit,err=99) ftype
!        write(iunit,err=99) nkn,nel,npr,nlv,nvar
!        write(iunit,err=99) date,time
!        write(iunit,err=99) title
!        write(iunit,err=99) femver
!        write(iunit,err=99) 
!
!        write(iunit,err=99) nen3v
!        write(iunit,err=99) ipev
!        write(iunit,err=99) ipv
!        write(iunit,err=99) iarv
!        write(iunit,err=99) iarnv
!        write(iunit,err=99) xgv
!        write(iunit,err=99) ygv
!        write(iunit,err=99) hm3v
!        write(iunit,err=99) hlv
!        write(iunit,err=99) ilhv
!        write(iunit,err=99) ilhkv
!        write(iunit,err=99) 
\end{verbatim}

For the meaning of the various parameters and variables please
see the legend below. |nvers| is a version number of the SHY files. Presently,
the version is 12.

The format of the data record depends is as follows:

\begin{verbatim}
!        write(iunit,iostat=ierr) dtime,ivar,n,m,lmax
!        if( lmax == 1 ) then
!          write(iunit,iostat=ierr) ( c(1,i),i=1,n*m )
!        else if( m == 1 ) then
!          write(iunit,iostat=ierr) (( c(l,i)
!     +                 ,l=1,il(i) )
!     +                 ,i=1,n )
!        else
!          stop 'error stop: impossible combination of m, lmax'
!        end if
\end{verbatim}

The meaning of the various parameters can be found here in the legend.
All variables and parameters are integer except where otherwise indicated.

\begin{verbatim}
! idshy         id to identify shy file (must be 1617)
! nvers         version of file format
! ftype         type of shy file (see below)
! nkn           total number of nodes
! nel           total number of elements
! npr           either 1 or 3, probably not used
! nlv           total nuber of vertical layers
! nvar          total number of variables in file
! date          reference date (integer, YYYYMMDD)
! time          reference time (integer, hhmmss)
! title         title of simulation (character*80)
! femver        version of shyfem (character*80)
!
! nen3v(3,nel)  element index
! ipev(nel)     external element number
! ipv(nkn)      external node number
! iarv(nel)     element type
! iarnv(nkn)    node type
! xgv(nkn)      x-coordinate (real)
! ygv(nkn)      y-coordinate (real)
! hm3v(3,nel)   depth at each vertex of elements (real)
! hlv(nlv)      layer depths (real, the bottom of each layer)
! ilhv(nel)     total number of layers for each element
! ilhkv(nkn)    total number of layers for each node
!
! dtime         time stamp (double precision, seconds)
! ivar          identification number of variable
! n             total number of horizontal values
! m             can be 1 (normal) or 3 for vertex values
! lmax          maximum number of layers given (1 for 2D)
!
! c(l,i)        data values (real)
! i,l           index for horizontal/vertical dimension
! il(i)         total number of layers for horizontal points
\end{verbatim}

Possible values for |ftype| are:

\begin{verbatim}
! 0             no type
! 1             hydro record (water levels, transports)
! 2             scalar values on nodes
! 3             scalar values on elements
\end{verbatim}

