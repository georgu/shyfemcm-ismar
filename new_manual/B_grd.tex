
%------------------------------------------------------------------------
%
%    Copyright (C) 1985-2020  Georg Umgiesser
%
%    This file is part of SHYFEM.
%
%    SHYFEM is free software: you can redistribute it and/or modify
%    it under the terms of the GNU General Public License as published by
%    the Free Software Foundation, either version 3 of the License, or
%    (at your option) any later version.
%
%    SHYFEM is distributed in the hope that it will be useful,
%    but WITHOUT ANY WARRANTY; without even the implied warranty of
%    MERCHANTABILITY or FITNESS FOR A PARTICULAR PURPOSE. See the
%    GNU General Public License for more details.
%
%    You should have received a copy of the GNU General Public License
%    along with SHYFEM. Please see the file COPYING in the main directory.
%    If not, see <http://www.gnu.org/licenses/>.
%
%    Contributions to this file can be found below in the revision log.
%
%------------------------------------------------------------------------
\label{grid}
The SHYFEM grid files have a defined format and file extension (.grd). Below the grd file format is expressed. 
The names used are described in the legend:
\begin{code}
n	item number (node, element, line)
t	type
d	depth
x,y	coordinates
ntot	number of following nodes
n1,n2	node numbers
\end{code}

The format of lines in input file is:

\begin{code}
comment:

0 [anything]

node:

1	n	t	x	y	[d]

element:

2	n	t	ntot	n1 n2 n3 ... 	[d]

line:

3	n	t	ntot	n1 n2 ...	[d]


\end{code}
Be careful about the following:
\begin{itemize}
\item Lines may be split at any point, except before optional argument.
\item d must not be on separate line.
\item If line is split, the continuation line(s) must start with a blank.
\item Blank lines can be inserted as needed
\item if d is not specified, -999. will be used (flag)
\item use t=0 if you do not know what to use
\item n must be unique for every item type 
\item item numbers need not be consecutive
\item the sequence of items is not important, 
nodes can be mixed with elements/lines
\item the minimum number of nodes for element items is 3
\item the minimum number of nodes for line items is 2
\item element items should have all nodes unique
\item line items with the same first and last node are considered closed
\end{itemize}

Examples of grd file format is given below.
\begin{code}
example 1 :

0 example of one line

1 11 0 10. 10.
1 12 0 20. 20.

3 7 0 2 11 12

#----------------

example 2 :

0 example of one element with continuation line

1 11 0 10. 10.
1 12 0 20. 20.
1 15 0 10. 20.

2 7 0 3 
   11 12 15

=============================================================

\end{code}

