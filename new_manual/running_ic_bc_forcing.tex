
%------------------------------------------------------------------------
%
%    Copyright (C) 1985-2020  Georg Umgiesser
%
%    This file is part of SHYFEM.
%
%    SHYFEM is free software: you can redistribute it and/or modify
%    it under the terms of the GNU General Public License as published by
%    the Free Software Foundation, either version 3 of the License, or
%    (at your option) any later version.
%
%    SHYFEM is distributed in the hope that it will be useful,
%    but WITHOUT ANY WARRANTY; without even the implied warranty of
%    MERCHANTABILITY or FITNESS FOR A PARTICULAR PURPOSE. See the
%    GNU General Public License for more details.
%
%    You should have received a copy of the GNU General Public License
%    along with SHYFEM. Please see the file COPYING in the main directory.
%    If not, see <http://www.gnu.org/licenses/>.
%
%    Contributions to this file can be found below in the revision log.
%
%------------------------------------------------------------------------
\subsection{Initial and boundary conditions}
%former bound_cond.tex

In order to have a more meaningfull simulation, we need to specify initial and
boundary conditions. 
Concerning initial conditions, spatially constant values can be set for water
 level, temperature, salinity, sediment or tracer concentration, 
in the |$para| section of the setup file. If spatially variable, eventually 3D fields are imposed as initial condition,
they have to be declared in |$name| section and properly formatted. Please refer to Appendix C.

In this section we will also deal with the open boundary
conditions, e.g., the conditions where the basin communicates with 
other water bodies (e.g., for lagoons it could be the inlets).

For every boundary condition one section |$bound| must be specified. Since
you can have more than one open boundary you must specify also the number
of your boundary, e.g., |$bound1|, |$bound2| etc. Inside every section
you can then specify the various parameters that characterize your boundary.

Basically there are two types of open boundary conditions. Either the water
level or the discharges (fluxes) can be specified. The parameter that
decides the type of boundary is |ibtyp|. A value of one indicates water
levels, instead a value of 2 or 3 indicates fluxes. If you specify
discharges entering at the border of the domain, |ibtyp| = 2 should be
specified. Otherwise, if there are internal sources in the basin then
|ibtyp| = 3 must be used. If you do not define this parameter, a value of 1
will be used and water levels will be specified.

The only compulsory parameter in this section is the list of boundary
nodes.  You do this with the parameter |kbound|. 
In the case of |ibtype| 1 or 2 at least two nodes must be
specified, in order to give an extension of the boundary. The numeration
of the boundary nodes must be consecutive and with the basin on its
left side when going along the boundary nodes.  In the case of |ibtyp|
= 3 even a single point can be given.

The boundary values are normally specified through a time series file. 
You give the name of the file that contains
the time series with the parameter |boundn|. 
An example with two boundaries can be found
 in \Fig\figref{example} where water levels are prescribed and 
the values are read from a |levels1.dat| file.

The imposed boundary values can be described through a simple sinus function by specifying  its parameters.
 An example of a water level boundary with a tide of
$\pm 70 cm$ and a period of 12 hours (semi-diurnal) is given below. Note thet |zref| gives the average water level of the
boundary. If you specify |ampli|=0 you get a constant boundary value
of |zref|.

%\begin{figure}[htbp]
%\begin{verbatim}
\begin{code}
$bound1
      ibtyp = 1   kbound = 23 25 28
      ampli = 0.70  period = 43200  phase = 10800  zref = 0.
$end
\end{code}
%\end{verbatim}
%\caption{
This is an example of a regular sinusoidal water level boundary.
The phase of 10800 (3 hours) makes sure that the simulation starts at
slack tide when the basin is completely full.
%}
%\label{fig:bound}
%\end{figure}

%%%%%%%%%%%%%%%%%%%%%%%%%%%%%%%%%%%%%%%%%%
\subsection{forcings}
%former wind_forcing.tex

\subsubsection{Wind forcing}
The wind and the mean sea level pressure can be prescribed by means of an external
 file with extension |fem| that can be either formatted or unformatted.
 Please see the section on file formats to write such a
file correctly.  The name of the file must be specified in the |str|
file in the section |name|, using the flag |wind|. For example:

\begin{code}
$name
...
wind = 'mywind.fem'
...
$end
\end{code}

Other important parameters to set in the |str| file, in the section
|para|, are |iwtype| and |itdrag|. The former specifies the wind stress
formulation, while the latter is used to prescribe a constant value of
the wind drag coefficient.

For more information see the appendix.


\subsubsection{Heat and Mass Fluxes}

%------------------------------------------------------------------------
%
%    Copyright (C) 1985-2020  Georg Umgiesser
%
%    This file is part of SHYFEM.
%
%    SHYFEM is free software: you can redistribute it and/or modify
%    it under the terms of the GNU General Public License as published by
%    the Free Software Foundation, either version 3 of the License, or
%    (at your option) any later version.
%
%    SHYFEM is distributed in the hope that it will be useful,
%    but WITHOUT ANY WARRANTY; without even the implied warranty of
%    MERCHANTABILITY or FITNESS FOR A PARTICULAR PURPOSE. See the
%    GNU General Public License for more details.
%
%    You should have received a copy of the GNU General Public License
%    along with SHYFEM. Please see the file COPYING in the main directory.
%    If not, see <http://www.gnu.org/licenses/>.
%
%    Contributions to this file can be found below in the revision log.
%
%------------------------------------------------------------------------
SHYFEM includes a module for the exchange of heat and mass (evaporation and rain) at the sea-atmosphere surface. This module must be activated only in case of a simulation with the calculation of temperature and salinity. The module is triggered by the |iheat| parameter, which decides which scheme to use. The humidity present in the air is prescribed by the physical variable defined with the |ihtype| parameter, while |isolp| decides the decay curve of solar radiation in the water column. The turbidity of the water can be specified with the |iwtyp| parameter and the depth of radiation decay (e-folding time), with the |hdecay| parameter. The fraction of radiation absorbed by the seabed is prescribed with the |botabs| parameter, while the albedo of the water 
surface is specified with the |albedo| parameter and, for temperatures below 4$^{\circ}$C, with |albed4|. Finally, the |ievap| parameter activates or deactivates the calculation of the water evaporation.

It is advisable to leave the default values of these parameters, or to change them following a calibration procedure. A detailed description can be found in Appendix \ref{parameter}.

Using the heat module a fem-file, or a time-series, with the surface forcing must be prescribed. This file must be specified in the str-file, in the |name| section, with the parameter |qflux |. The file must contain the following variables: solar radiation, air temperature, humidity and cloud cover. Finally, in the same section, it is possible to prescribe a file with the precipitation, through the parameter |rain |.
 








