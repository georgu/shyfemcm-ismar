\section{Parameter list for the post processing routines}

The format of the parameter input file is the same as the one for
the main routine. Please see this section for more information
on the format of the parameter input file.

Some sections of the parameter input file are identical to the
sections used in the main routine. For easier reference we will
repeat the possible parameters of these section here.


\subsection{Section {\tt \$title}}

This section must be always the first section in the parameter input file.
It contains only three lines. An example has been given already in
\Fig\ref{fig:titleexample}.

The only difference with respect to the {\tt \$para} section of the main
routine is the first line. Here any description of the output can be used.
It is just a way to label the parameter file.  The other two lines with
the name of simulation and the basin are used to open the files needed
for plotting.


\subsection{Section {\tt \$para}}


These parameters set generic values for the plot.

Some of the parameters set coordinates in the plot. For example, the
values |x0, y0| and |x1, y1| indicate the actual plotting area, which can
be bigger or smaller than the extension of the numerical grid.

Normally, values have to be in meters (the same as the coordinates in the
numerical grid). However, also relative coordinates can be used. If all
values given are in the range between -1 and +2, then these values
are interpreted as relative coordinates. Therefore, $x$ coordinates of
0 indicate the left border, and 1 the right border. The upper left quarter
of the domain can be chosen with (|x0, y0|) = (0,0.5) and
(|x1, y1|) = (0.5,1).

\descrpitem{|x0\\y0|}
\descrptext{%
Lower left corner of the plotting area.
(Default is whole area)
}
\par
\descrpitem{|x1\\y1|}
\descrptext{%
Upper right corner of the plotting area.
(Default is whole area)
}
\par

The next values give the position, where the legend (scale bar and
true north) is plotted. This legend will only be plotted if
the coordinates are not geographical (lat/lon) but cartesian.

\descrpitem{|x0leg\\y0leg|}
\descrptext{%
Lower left corner of the area
where the legend is plotted.
}
\par
\descrpitem{|x1leg\\y1leg|}
\descrptext{%
Upper right corner of the area where 
the legend (north and scale) is plotted.
}
\par

\descrpitem{|lblank|}
\descrptext{%
The legend is plotted over a white rectangle.
Sometimes this blanking is not desireable.
If you do not want to have a white box below the
legend set |lblank| to 0. (Default 1)
}
\par

\descrpitem{|cislnd|}
\descrptext{%
It is possible to plot all islands in gray color.
Setting |cislnd| to a value between 0 (black) and 
1 (white) will achieve this. A negative value
will not fill islands with gray color.
(Default -1)
}
\par

\descrpitem{|dgray|}
\descrptext{%
It is possible to plot all dry areas in gray color.
Setting |dgray| to a value between 0 (black) and 
1 (white) will achieve this. A negative value
will not fill dry areas with gray color.
(Default -1)
}
\par

\descrpitem{|hgray|}
\descrptext{%
Whereas |dgray| is normally only coloring
elements that are dry, you can also color elements
shallower than a given depth |hgray|. E.g., a value
for |hgray| of -0.5 will plot in gray all
elements with depth lower than -0.5 m (salt
marshes). (Default -10000)
}
\par

\descrpitem{|dxygrd|}
\descrptext{%
Grid size if the results are interpolated on
a regular grid. A value of 0 does
not use a regular grid but the original
finite element grid for plotting. (Default 0)
}
\par
\descrpitem{|typls|}
\descrptext{%
Typical length scale to be used when scaling
velocity or transport arrows. If |dxygrd| is
given this length is used and |typls| is not used.
If not given it is computed from the basin
parameters. (Default 0)
}
\par
\descrpitem{|typlsf|}
\descrptext{%
Additional factor to be used with |typls| to
determine the length of the maximum or
reference vector. This is the easiest way
to scale the velocitiy arrows 
with an overall factor. (Default 1)
}
\par
\descrpitem{|velref|}
\descrptext{%
Reference value to be used when scaling arrows.
If given, a vector with this value will have a length
of |typls|*|typlsf| on the map, or, in case
|dxygrd| is given, |dxygrd|*|typlsf|. If not set
the maximum value of the velocity/transport
will be used as |velref|. (Default 0)
}
\par
\descrpitem{|velmin|}
\descrptext{%
Minimum value for which an arrow will be plotted.
With this value you can eliminate small arrows
in low dynamic areas. (Default 0)
}
\par
\descrpitem{|velmax|}
\descrptext{%
Maximum value for which an arrow will be plotted.
With this value you can eliminate arrows that are
too big. This is usefull if you would like to study an
area with low current speed but but the adjacent area has high current speeds that would overplot the area.
(Default -1, no limitation)
}
\par

\descrpitem{|isphe|}
\descrptext{%
If 0, a cartesian coordinate system is used,
If 1, the coordinates are in the spherical 
system (lat/lon). Among other, this
indicates that the $x$-coordinates will be multiplied
by a factor that accounts for the visual deformation
using lat/lon coordinates.
The default is -1, which means that the 
type of coordinate system will 
be determined automatically. (Default -1)
}
\par
\descrpitem{|reggrd|}
\descrptext{%
If different from 0 it plots a regular grid over
the plot for geographical reference. The value of
|reggrd| gives the spacing of the regular grid lines.
The units must be according to the units used for
the coordinates. With value of -1 the regular grid is
determined automatically. (Default -1)
}
\par
\descrpitem{|regdst|}
\descrptext{%
This value gives the number of intervals
that are used to sub-divide the grid given by
|reggrd| with a black and white scale around
the plot. If 0 it tries to determine automatically
the sub-intervals (2 or 4). A value of -1 does
not plot the subgrid scale. (Default 0)
}
\par
\descrpitem{|reggry|}
\descrptext{%
If plotting the regular overlay grid this gives
the gray value used for the grid. 0 is black, and
1 is white. A value of 1 does not plot the
overlay grid, but still writes the labels. 
(Default 1)
}
\par

\descrpitem{|bndlin|}
\descrptext{%
Name of file that gives the boundary line
that is not part of the finite element domain.
The file must be in BND format. You can use
the program grd2bnd.pl to create the file
from a GRD file.
(Default is no file)
}
\par

\descrpitem{|ioverl|}
\descrptext{%
Create overlay of velocity vectors on scalar value.
With the value of 0 no overlay is created, 1
creates an overlay with the velocity speed.
The value of 2 overlays vertical velocities,
3 water levels and 4 overlays bathymetry.(Default 0)
}
\par
\descrpitem{|inorm|}
\descrptext{%
Normally the horizontal velocities are plotted
in scale. The value of |inorm| can change this
behavior. A value of 1 normalizes velocity vectors
(all vectors are the same length), whereas 2
scales from a given minimum velocity |velmin|.
Finally, the value of 3 uses a logarithmic scale.
(Default 0)
}
\par

The next parameters give the choice to selectively avoid to plot areas
of the basin and to apply different gray tones for the boundary and
net lines when plotting the basin.
Please remember that when working with gray tones the value should
be between 0 (black) and 1 (white).

\descrpitem{|ianopl|}
\descrptext{%
Area code for which no plot has to be produced. Normally 
the whole basin is plotted, but with this parameter some
areas can be excluded. (Default -1)
}
\par
\descrpitem{|bgray|}
\descrptext{%
Gray value used for the finite element grid when plotting
the bathymetry. (Default 0.8)
}
\par
\descrpitem{|bbgray|}
\descrptext{%
Gray value used for the boundary of the finite element grid.
(Default 0)
}
\par
\descrpitem{|bsgray|}
\descrptext{%
Gray value used to plot the finite element grid over
a scalar or velocity plot. This is basically useful
for debugging reasons. The default is to not plot
the grid (Default -1.0)
}
\par

\subsection{Section {\tt \$color}}


The next parameters deal with the definition of the colors
to be used in the plot. A color bar is plotted too.

\descrpitem{|icolor|}
\descrptext{%
Flag that determines the type of color table
to be used:
\begin{description}
\item 0 gray scale
\item 1 HSB color table 
\item 2 from white to blue
\item 3 from white to red
\item 4 from blue over white to red 
\item 5 from blue over black to red
\item 6 and 7 indicate non-linear HSB color tables.
(Default 0)
\end{description}
}
\par

\descrpitem{|colfil|}
\descrptext{%
A color table can also be read from file. An example
of the format can be found in directory |femplot/color|
in the file |colormap.dat|. The variable |colfil|
indicates the file where the color table is being
read from. The default is not to read a color table file.
}
\par
\descrpitem{|coltab|}
\descrptext{%
If a color table file has been read then the variable
|coltab| indicates the name of the color table that
is going to be used. The default is to not use any
of the color tables if no name is specified.
}
\par

\descrpitem{|isoval|}
\descrptext{%
Array that defines the values for the isolines
and colors that are to be plotted. Values given must
be in the unit of the variable that will be plotted,
i.e., meters for water levels etc. 
}
\par
\descrpitem{|color|}
\descrptext{%
Array that gives the color indices for the
plotting color to be used. Ranges are from
0 to 1. The type of the color depends on the 
variable |icolor|. For the gray scale table
0 represents black and 1 white. Values in between
correspond to tones of gray. For the HSB color table
going from 0 to 1 gives the color of the rainbow.
There must be one more value in |color| than in |isoval|.
The first color in |color| refers to values less
than |isoval(1)|, the second color in |color| to
values between |isoval(1)| and |isoval(2)|. The last
color in |color| refers to values greater than the last
value in |isoval|.
}
\par

\descrpitem{|x0col\\y0col|}
\descrptext{%
Lower left corner of the area where the
color bar is plotted.
}
\par
\descrpitem{|x1col\\y1col|}
\descrptext{%
Upper right corner of the area where the
color bar is plotted.
}
\par

\descrpitem{|cblank|}
\descrptext{%
The color bar is plotted over a white rectangle.
Sometimes this blanking is not desireable.
If you do not want to have a white box below the
legend set |cblank| to 0. (Default 1)
}
\par

\descrpitem{|faccol|}
\descrptext{%
Factor for the values that are written to the 
color bar legend. This enables you, e.g., to give water level
results in mm (|faccol = 1000|). (Default 1)
}
\par
\descrpitem{|ndccol|}
\descrptext{%
Decimals after the decimal point for the values
written to the color bar legend. Use the value |-1|
to not write the decimal point. A value of 0 automatically
computes the number of decimals needed. (Default 0)
}
\par
\descrpitem{|legcol|}
\descrptext{%
Text for the description of the color bar. This text
is written above the color bar.
}
\par

It is not necessary to give all values for isolines and colors above.
A faster way is to give only the minimum and maximum values and fix
the number of isovalues to be used.

\descrpitem{|niso|}
\descrptext{%
Total number of isolines to use. (Default is |nisodf|)
}
\par
\descrpitem{|nisodf|}
\descrptext{%
Default number of isolines to use. (Default 5)
}
\par
\descrpitem{|colmin\\colmax|}
\descrptext{%
Minimum and maximum color index used. Defaults are
0.1 and 0.9 respectively. The value of |colmax| can
be smaller than |colmin| which inverts the color
index used.
}
\par
\descrpitem{|valmin\\valmax|}
\descrptext{%
Minimum and maximum value for isovalues to be used.
There is no default.
}
\par
\descrpitem{|rfiso|}
\descrptext{%
Defines function to be used to compute intermediate
values between |valmin| and |valmax|. If 0 or 1
the values are linearlily interpolated. Else they
are computed by $y=x^n$ where $n$ is |rfiso|
and $x=\frac{v-v_{min}}{v_{max}-v{min}}$. Values
for |rfiso| greater than 0 capture higher detail in
the lower values, whereas values less than 1 do
the opposite.
(Default 0)
}
\par
\descrpitem{|ipllog|}
\descrptext{%
Indicates the usage of a logarithmic color scale.
The possible values are 0-3. The value of 0
indicates not to use a logarithmic scale.
If 1, the values of
the scale are 1,10,100,etc., if 2 the values
1,2,10,20,100,etc. are used, and for 3 the values
are 1,2,5,10,20,50,100,etc. (Default 0)
}
\par
\descrpitem{|dval|}
\descrptext{%
Difference of values between isolines. If this
value is greater then 0 the values for isolines 
and the total number of isolines are computed 
automatically using also |valmin| and |valmax|. 
(Default 0)
}
\par

Since there is a great choice of combinations between the parameters,
the following rules show how the values for colors and isolines are determined.

If colors are given in array |color|, they are used, else |colmin| and
|colmax| or their respective defaults are used to determine the color bar.
If |isoval| is given it is used, else |valmin| and |valmax| are used.
If |valmin| and |valmax| are not given they are computed every time
for each plot and the minimum and maximum values in the basin are used.
In any case, if |isoval| is specified the total number of isovalues
is known and |niso| is ignored. However, if |isoval| is not given
then first |dval| is used to decide how many isovalues to plot, and
if |dval| is 0 then the |niso| and, finally, |nisodf| are used.

Other parameters that can be changed are the following.

\descrpitem{|nisomx|}
\descrptext{%
Maximum for |niso| allowed. This is especially useful
when the value for |niso| is determined automatically.
It avoids you to plot 1000 isolines due to wrong settings
of |dval|. However, if you want to use 50 isovalues
then just set |niso| and |nisomx| to 50. (Default 20)
}
\par
\descrpitem{|nctick|}
\descrptext{%
Number of values to be written in color bar. If |niso| is high
the labels on the color bar become unreadable. Therefore
you can use |nctick| to write only some of the values to
the color bar. For example, if |valmin| is 0 and |valmax| is
5 and you use many isolines, then setting |nctick| to 6 would
give you labels at values 0,1,2,3,4,5. If |nctick| is 0
then all lables are written. (Default 0)
}
\par
\descrpitem{|isolin|}
\descrptext{%
Normally the isolines are not drawn on the plot, just
the colors are used to show the value in the different
parts of the plot. A value different from 0 plots also 
the isolines. In this case |isolin| gives the number of
isolines to be plotted. A good choice is to make this
equal to |nctick|, so that the isolines correspond to the 
values	written on the colorbar. For compatibility, a value of
1 plots all isolines. (Default 0)
}
\par
\descrpitem{|isoinp|}
\descrptext{%
Normally inside elements the values are interpolated.
Sometimes it is usefull to just plot the value of the
node without interpolation inside the element. This can
be accomplished by setting |isoinp=0|. (Default 1)
}
\par

\subsection{Section {\tt \$arrow}}


The next parameters deal with the reference arrow that is plotted
in a legend. The arrow regards the plots where the velocity or
the transport is plotted.

\descrpitem{|x0arr\\y0arr|}
\descrptext{%
Lower left corner of the area where the
reference arrow is plotted.
}
\par
\descrpitem{|x1arr\\y1arr|}
\descrptext{%
Upper right corner of the area where the
reference arrow is plotted.
}
\par

\descrpitem{|ablank|}
\descrptext{%
The arrow legend is plotted over a white rectangle.
Sometimes this blanking is not desireable.
If you do not want to have a white box below the
legend set |ablank| to 0. (Default 1)
}
\par

\descrpitem{|facvel|}
\descrptext{%
Factor for the value that is written to the 
arrow legend for the velocity.
This enables you, e.g., to give 
velocities in mm/s (|facvel = 1000|). (Default 1)
}
\par
\descrpitem{|ndcvel|}
\descrptext{%
Decimals after the decimal point for the values
written to the arrow legend. 
Use the value |-1|
to not write the decimal point. (Default 2)
}
\par
\descrpitem{|legvel|}
\descrptext{%
Text for the description of the arrow legend.
This text is written above the arrow.
}
\par
\descrpitem{|arrvel|}
\descrptext{%
Length of arrow in legend (in velocity
units). If not given the arrow length will be computed
automatically. (Default 0)
}
\par
\descrpitem{|sclvel|}
\descrptext{%
Additional factor to be used for the arrow
in the legend. When the arrow length will be
computed automatically, this parameter gives
the possibility to change the length of the
reference vector. This is an easy way
to scale the velocitiy arrow
with an overall factor. Not used if
|arrvel| is given. (Default 1)
}
\par

\subsection{Section {\tt \$legend}}
In this section, annotations in the plots can be given. The
section consists of a series of lines that must contain the 
following information:

The first value is a keyword that specifies what has to be plotted.
Possible values are |text|, |line|, |vect|, |rect|, |circ|  and also
|wid| and |col|. These correspond to different types of information
that is inserted into the plot such as text, line, vector, rectangle
or circle (filled or just outline). Moreover, the color and
line width of the pen can be controlled with |wid| and |col|.

In case of |text| the starting position (lower left corner) is given,
then the point size of the font and the text that is inserted. |line|
needs the starting and end point of the line. The same with |vect|,
but in this case also the relative tip size must be given as a final
parameter. |rect| needs the coordinates of the lower left corner and 
upper right corner of the rectangle. It also needs the color used for
the filling of the rectangle (0-1) or the flag -1 which only draws the
outline of the rectangle without filling it. |circ| needs the center
point and the radius and a fill color (see rectangle). Finally |wid| needs 
the relative width of the line and |col| the stroke color used when plotting
lines.

A small example of an annotation that explains the above parameters
would be:

\begin{verbatim}
$legend
text  30500 11800     15  'Chioggia'   #text, 15pt
line  30500 11800 35000 15000          #line
vect  30500 11800 35000 15000 0.1      #arrow, tipsize 0.1
rect  30500 11800 35000 15000 0.1      #rectangle, fill color 0.1
rect  30500 11800 35000 15000 -1       #rectangle (outline, no fill)
circ  30500 11800 5000 -1              #circle (outline, no fill)
wid   5                                #set line width to 5
col   0.5                              #set color to 0.5
$end
\end{verbatim}

There is also an old way to specify the legend that does not use
keywords. However, this way is deprecated and unsupported and is therefore
not described anymore in this manual.

\subsection{Section {\tt \$legvar}}


In this section variable fields like the date and wind vectors
may be inserted into the plot.

A time and date can be assigned to the simulation results. These values
refer to the time 0 of the FEM model. The format for the date is
YYYYMMDD and for the time HHMMSS. 
You can also give a time zone if your time is not referring to 
GMT but to another time zone such as MET. Please note that you have to give
this information only if the simulation does not contain it already.
Normally, this information is already assigned during the simulation runs.

\descrpitem{|date|}
\descrptext{%
The real date corresponding to time 0. (Default 0)
}
\par
\descrpitem{|time|}
\descrptext{%
The real time corresponding to time 0. (Default 0)
}
\par
\descrpitem{|tz|}
\descrptext{%
The time zone you are in. This is 0 for GMT, 1 for MET
and 2 for MEST (MET summer time). (Default 0)
}
\par
\descrpitem{|tzshow|}
\descrptext{%
The time zone you want to show your results. If your
time zone is GMT (0) and you want to show the results
referred to MET (+1) set this to +1. Please note that
you have to set this variable only if you want to
show results in a different time zone than the one
given in |tz|. (Default 0)
}
\par

The information of date and time may be written to the plot. This
is done with the following parameters.

\descrpitem{|xdate\\ydate|}
\descrptext{%
Starting point for the date text (lower left corner).
}
\par
\descrpitem{|sdate|}
\descrptext{%
Point size of the text. (Default 18)
}
\par
\descrpitem{|idate|}
\descrptext{%
Output mode. If 0 no date is written to the
plot, else the date and time is written. (Default 0)
}
\par

Wind data can be used to insert a wind vector into the figure.
This can be useful in the case of varying wind conditions as 
the exact speed and direction of the time frame represented 
in the plot will be shown.

Since only one wind vector can be plotted, the wind data must consist
of one value for each time. The same ASCII file that is used
in the STR file can be used.

\descrpitem{|xwind\\ywind|}
\descrptext{%
Starting point where the wind arrow is plotted.
}
\par
\descrpitem{|iwtype|}
\descrptext{%
Type of wind data. The same as the one in the
STR file. If this parameter is 0 then no
wind vector is plotted. (Default 0)
}
\par
\descrpitem{|lwwind|}
\descrptext{%
Line width of the wind vector. (Default 0.1)
}
\par
\descrpitem{|scwind|}
\descrptext{%
Scaling parameter of the wind vector. This depends
on the size of your plot. If your wind is 10 m/s
and you want the vector to strech over a distance
of 5 km on the plot then you have to choose
the value of 500 (10*500=5000) for |scwind|.
(Default 1)
}
\par
\descrpitem{|wfile|}
\descrptext{%
Name of the file containing the wind data. This 
may be the same file than the one used in the
STR file to run the program.
}
\par

The wind vector is also given a text legend with the speed of the
wind written out. The next parameters decide where and how this information
is set.

\descrpitem{|xtwind\\ytwind|}
\descrptext{%
Starting point for the legend text (lower left corner).
}
\par
\descrpitem{|stwind|}
\descrptext{%
Point size of the text. (Default 18)
}
\par
\descrpitem{|wtext|}
\descrptext{%
Text used for the legend (Default 'Wind speed')
}
\par
\descrpitem{|wunit|}
\descrptext{%
Unit for the wind speed (Default 'm/s')
}
\par

%\subsection{Section {\tt \$name}}
%\input{S_name.tex}

