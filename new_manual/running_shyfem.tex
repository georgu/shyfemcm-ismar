
%------------------------------------------------------------------------
%
%    Copyright (C) 1985-2020  Georg Umgiesser
%
%    This file is part of SHYFEM.
%
%    SHYFEM is free software: you can redistribute it and/or modify
%    it under the terms of the GNU General Public License as published by
%    the Free Software Foundation, either version 3 of the License, or
%    (at your option) any later version.
%
%    SHYFEM is distributed in the hope that it will be useful,
%    but WITHOUT ANY WARRANTY; without even the implied warranty of
%    MERCHANTABILITY or FITNESS FOR A PARTICULAR PURPOSE. See the
%    GNU General Public License for more details.
%
%    You should have received a copy of the GNU General Public License
%    along with SHYFEM. Please see the file COPYING in the main directory.
%    If not, see <http://www.gnu.org/licenses/>.
%
%    Contributions to this file can be found below in the revision log.
%
%------------------------------------------------------------------------

In the following an overview is given on running the model \shy{}. In the SHYFEM code directory you can find a directory called |examples|.  In it you can find examples to build from simple 2D to more complex 3D setup files. All explanations are
provided in the README file and in the wiki at the following
link:
\\
\\
\underline{\texttt{https://github.com/georgu/shyfemcm-ismar/wiki/Test-cases}}
\\
\\
The
model needs a parameter input file in ASCII format, with extension |str|,
that is read on standard input. Moreover, it needs some external files
that are specified in this parameter input file. The model produces
several external files with the results of the simulation. Again, the
name of this files can be influenced by the parameter input file.
Once the str file (e.g., param.str) is made, the following line starts the simulation:

\begin{verbatim}
          shyfem param.str
\end{verbatim}
