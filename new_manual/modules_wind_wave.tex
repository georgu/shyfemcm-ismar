 SHYFEM can be coupled with the spectral wind wave unstructured
 model WWMIII (Wave Watch III Model) or it can compute wave characteristics using empirical
 prediction equations.

 This empirical wave module is used to calculate the wave height 
 and period from wind speed, fetch and depth using the empirical prediction
equations for shallow water \cite{shoreprot:84}.

 WWMIII is not provided in the SHYFEM distribution.
 The coupling of SHYFEM with WWMIII is done through the FIFO PIPE 
 mechanism. The numerical mesh needs to be converted to the GR3 
 format using the bas2wwm program. WWMIII needs its own input 
 parameter file (wwminput.nml). The use of the coupled SHYFEM-WWMIII
 model requires additional software which is not described here.

 The wave module writes in the WAV file the following outputs:
 \begin{itemize}
 \item significant wave height [m], variable 231
 \item mean wave period [s], variable 232
 \item mean wave direction [deg], variable 233
 \end{itemize}

 The time step and start time for writing to file WAV 
 are defined by the parameters |idtwav| and |itmwav| in the |waves|
 section. These parameter are the same used for writting tracer
 concentration, salinity and water temperature. If |idtwav| is not
 defined, then the wave module does not write any results. The wave 
 results can be plotted using |plots -wav|.

 In case of SHYFEM-WWMIII coupling, several variables are exchanged 
 between the two models:
 \begin{itemize}
 \item SHYFEM sends to WWMIII:
  \begin{itemize}
   \item surface velocities
   \item water level
   \item bathymetry and number of vertical layers
   \item 3D layer depths
   \item wind components$^{**}$
  \end{itemize}
 \item SHYFEM reads from WWMIII:
  \begin{itemize}
   \item gradient of the radiation stresses
   \item significant wave height
   \item mean period
   \item significant wave direction
   \item wave supported stress
   \item peak period
   \item wave length
   \item orbital velocity
   \item stokes velocities
   \item wind drag coefficient
   \item wave pressure
   \item wave dissipation
   \item wind components$^{**}$
 \end{itemize}
 \end{itemize}

 $^{**}$Wind could be either read from SHYFEM or WWMIII, see parameter 
 |iwave| in Appendix C.
 For more information about WWMIII and its coupling with SHYFEM please refer
 to Roland et al. \cite{roland:coupled09} and Ferrarin et al. 
 \cite{ferrarin:morpho08}.
