All forcing files needed by SHYFEM are either simple time series files
or FEM files. The format of the FEM files can be found in appendix B.2 in the file src/utils/generic/femfile.f90.

\subsection{nc2fem}
Here we describe how a FEM file can be produced starting from a NETCDF
file. This should be the normal procedure, since many data is nowadays
distributed by NETCDF (.nc) files.

The program to be used is called nc2fem. Is installed only if you have
enabled NETCDF support in the Rules.make file. How to do this can be
found in HOWTO-NETCDF.txt.  Please note that nc2fem has an online help
that can be seen with "nc2fem -h".

Running nc2fem on a NETCDF file will give you only basic information.
More information can be found running "nc2fem -varinfo file.nc" where
file.nc is your NETCDF file. In the example given here it will an output
such as:\\

\begin{tabular}{lccccccl} \hline
  id & natts & ndim &  name / description \\
 \hline
  1   &  4   &  1      &  time  $/$ time \\
  2    & 4   &  1      & lon $/$ longitude \\
  3   &   4   &   1    & lat $/$ latitude \\
  4   &   7   &   3    & msl $/$ air pressure at mean sea level \\
  5   &   6   &   3   & 10v $/$ 10 metre V wind component \\
  6   &   6   &   3    &10u $/$ 10 metre U wind component \\
 \hline
\end{tabular}


If you do not want to use nc2fem to get this information, you can also
run "ncdump -h file.nc" which will give you a comprehensive list of all
the content in the file.

Since in this example we want to create a file for the wind forcing,
we will need the wind velocity and the atmospheric pressure. This can
be achieved by running "nc2fem -vars '10u,10v,msl' file.nc". It tells
nc2fem to extract the variables 10u, 10v, and msl from the file and
write them to the fem file "out.fem". If everything goes well you are
already finished. However, it is a good idea to rename the output file
"out.fem" to something more descriptive such as "wind.fem".

If the program complains that it cannot understand the meaning of the
some of the variables, you will have to specify the description on the
command line.  To do this first run "nc2fem -list" to find the acronyms
of the variable you have to specify, and then put them on the command
line. In the case above this would be "nc2fem -vars '10u,10v,msl' -descrp
'wind,wind,airp' file.nc". This tells the program that the first two
variables contain the wind, and the last one the atmospheric pressure.
You only have to specify the description of the variables that are not
recognized by nc2fem. So, if it complains about not knowing what msl is
you can specify the description as "-descrp ',,airp'".

You can do other things like limit the area of the domain that will be
extracted and written into out.fem or change the original values using
a factor or offset.

SHYFEM uses various files for forcing. For atmospheric forcing the
files to be given are wind, heat, and rain. The wind file has to contain
'wind,wind,airp' in that order, the heat file needs 'srad,airt,rhum,cc'
where the variables are respectively solar radiation, air temperature,
relative humidity, and cloud cover'. You can also use specific humidity by
specifying shum instead of rhum. Finally the rain file needs the variable
rain. This can be a bit tricky, because the value to be inserted is in
mm/day. If you are unsure about the conversion from the NETCDF file,
after creating rain.fem you can run it though femelab with the following
command line option: "femelab -checkrain rain.fem". This will give you an
idea of the entity of the rain you have produced. If this is different
from what you are expecting, you will have to revise the conversion
factor for the rain variable.

For the hydrodynamic variables you will probably need boundary and
initial conditions. Use zeta for water level, vel for velocities, and
salt and temp for salinity and temperature respectively. Again, please
check "nc2fem -list" for the right acronym to use for your variable.
Please remember that the last variables are 3D variables, so you will
have to specify them also in the vertical. If you only have 2D variables,
or the data does not cover the whole water column, SHYFEM will use the
last available data in the vertical and extend it down to the bottom. For
2D data this means that the whole water column will be homogeneous.



SHYFEM provides also routines to inquire and elaborate fem files. Below the list with the description of their full
functionalities.
\subsection{femelab}
|femelab |is the routine for elaborating FEM files. It also returns
information on a FEM file.

\begin{verbatim}
 Usage: femelab [-h|-help] [-options] fem-file
  returns info on or elaborates a shyfem file
  options:
   -h|-help            this help screen
   -v|-version         version of routine
  general options
   -info               only give info on header
   -verbose            be more verbose, write time records
   -quiet              do not write header information
   -silent             do not write anything
   -write              write min/max of records
  time options
   -tmin time          only process starting from time
   -tmax time          only process up to time
   -inclusive          output includes whole time period given
      time is either YYYY-MM-DD[::hh[:mm[:ss]]]
      or integer for relative time
  output options
   -out                writes new file
   -outformat form     output format
      possible output formats are: shy|gis|fem|nc|off (Default native)
      not all formats are available for all file types
   -catmode cmode      concatenation mode for handeling more files
      possible values for cmode are: -1,0,+1 (Default 0)
      -1: all of first file, then remaining of second
       0: simply concatenate files
      +1: first file until start of second, then all of second
   -proj projection    projection of coordinates
      projection is string consisting of mode,proj,params
      mode: +1: cart to geo,  -1: geo to cart
      proj: 1:GB, 2:UTM, 3:CPP
  extract options
   -split              split file for variables
   -check period       checks data over period
    period can be all,year,month,week,day,none
   -checkdt            check for change of time step
   -checkrain          check for yearly rain (if file contains rain)
   -coord coord        extract coordinate
      coord is x,y of point to extract
  specific FEM file options
   -condense           condense file data into one node
   -chform             change output format form/unform of FEM file
   -grd                write GRD file from data in FEM file
   -nodei node         extract internal node
      node is internal numbering in fem file or ix,iy of regular grid
   -newstring sstring  substitute string description in fem-file
      sstring is comma separated strings, empty for no change
   -facts fstring      apply factors to data in fem-file
      fstring is comma separated factors, empty for no change
  regular output file options
   -reg rstring        regular interpolation
   -resample bounds    resample regular grid
   -regexpand iexp     expand regular grid
      rstring is: dx[,dy[,x0,y0,x1,y1]]
      if only dx is given -> dy=dx
      if only dx,dy are given -> bounds computed
      bounds is: x0,y0,x1,y1
      iexp>0 expands iexp cells, =0 whole grid
      resample should be used with regexpand
\end{verbatim}

\subsection{femtide}
This program performs tidal analysis of water levels in fem format
The considered constituents are: 
\begin{itemize}
\item semi-diurnal: M2, S2, N2, K2, NU2, MU2, L2, T2
\item diurnal:      K1, O1, P1, Q1, J1, OO1, S1
\item long period:  MF, MM, SSA, MSM, MSM, SA
\end{itemize}

It produces two files:
\begin{itemize}
\item out.tid.fem  containing the tidal signal
\item out.res.fem  containing the residual signal
\end{itemize}

\begin{verbatim}
 Usage: femtide [-h|-help] [-options] fem-file
  performs tidal analysis on fem-file
  options:
   -h|-help      this help screen
   -v|-version   version of routine
  options in/output
   -verb         be more verbose
   -silent       be silent
   -quiet        do not write time records
  additional options
   -lat val      latitude for nodal correction (default 45)
\end{verbatim}

\subsection{tselab}
Routine for elaborating time-series files. It also returns
information on a time-series file.

\begin{verbatim}
 Usage: tselab [-h|-help] [-options] time-series-file
  returns info on or elaborates a shyfem file
  options:
   -h|-help          this help screen
   -v|-version       version of routine
  general options
   -info             only give info on header
   -verbose          be more verbose, write time records
   -quiet            do not write header information
   -silent           do not write anything
   -write            write min/max of records
  time options
   -tmin time        only process starting from time
   -tmax time        only process up to time
   -inclusive        output includes whole time period given
      time is either YYYY-MM-DD[::hh[:mm[:ss]]]
      or integer for relative time
  output options
   -out              writes new file
   -outformat form   output format
      possible output formats are: shy|gis|fem|nc|off (Default native)
      not all formats are available for all file types
   -catmode cmode    concatenation mode for handeling more files
      possible values for cmode are: -1,0,+1 (Default 0)
      -1: all of first file, then remaining of second
       0: simply concatenate files
      +1: first file until start of second, then all of second
   -proj projection  projection of coordinates
      projection is string consisting of mode,proj,params
      mode: +1: cart to geo,  -1: geo to cart
      proj: 1:GB, 2:UTM, 3:CPP
  extract options
   -split            split file for variables
   -check period     checks data over period
    period can be all,year,month,week,day,none
   -checkdt          check for change of time step
   -checkrain        check for yearly rain (if file contains rain)
  time series options
   -convert          convert time column to ISO string
   -date0 string     reference date for conversion of time column
\end{verbatim}

