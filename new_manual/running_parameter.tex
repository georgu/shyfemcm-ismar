
%------------------------------------------------------------------------
%
%    Copyright (C) 1985-2020  Georg Umgiesser
%
%    This file is part of SHYFEM.
%
%    SHYFEM is free software: you can redistribute it and/or modify
%    it under the terms of the GNU General Public License as published by
%    the Free Software Foundation, either version 3 of the License, or
%    (at your option) any later version.
%
%    SHYFEM is distributed in the hope that it will be useful,
%    but WITHOUT ANY WARRANTY; without even the implied warranty of
%    MERCHANTABILITY or FITNESS FOR A PARTICULAR PURPOSE. See the
%    GNU General Public License for more details.
%
%    You should have received a copy of the GNU General Public License
%    along with SHYFEM. Please see the file COPYING in the main directory.
%    If not, see <http://www.gnu.org/licenses/>.
%
%    Contributions to this file can be found below in the revision log.
%
%------------------------------------------------------------------------

In the following the basic structure of the input file is described. Refer to Appendix C 
for the full list of parameters that can be set. The description will provide indication on how to run a simple simulation.
In the following chapter a deeper insight in how to prepare a more complex input file will be provided.


\subsection{Structure}

The input parameter file is the file that guides the program performance. It
contains all the necessary information for the main routine to execute
the model. Nearly all parameters that can be given have a default value
which is used when the parameter is not listed in the file. Only some
time parameters are compulsory and must be present in the file.

The file format looks much like a namelist format, but is
not dependent on the compiler used. Values of parameters are given
in the form :  
|name = value|  or  |name = 'text'|.  If |name|
is an array the following format is used : 
\begin{verbatim}
          name = value1 , value2, ... valueN
\end{verbatim}
The list can continue on the following lines. Blanks before and after
the equal sign are ignored. More than one parameter can be present
in one line. Blank, tab and comma can be used as separators.

Parameters, arrays and data  are delimited in sections.
A section starts with the character {\tt \$} followed by a keyword and
ends with {\tt \$end}. The {\tt \$keyword} and {\tt \$end} must not
contain any blank characters and must be the first non blank characters
in the line. Other characters following the keyword on the same line
separated by a valid separator are ignored.

Several sections of data may be present in the input parameter file.
Further ahead, all possible sections are presented together with
 the parameters prone to be specified.
The sequence in
which the sections appear is of no importance. However, 
the section |\$title| must be the first, which determines 
the name of the simulation, the basin file and provides a one line description of the simulation.

Lines outside of the sections are ignored. This gives
the possibility to comment the parameter input file.

\begin{figure}[htbp]
\begin{alltt}
\input{basic.str}
\end{alltt}
\caption{Example of a basic parameter input file ({\tt STR} file)}
\label{fig:basic}
\end{figure}
%%%%%%%%%%%%%%%%%%%%%%%%%%%%%%%%%%%%%%%%%%%%%
% former basic_minimal.tex

%\importstr{basic}
%{Example of a basic parameter input file ({\tt str} file)}

A basic version of an |str| file can be found in \ref{fig:basic} in which 
only the compulsory parameters have been inserted (in fact, this str does not do anything).
These are:

\begin{itemize}

\item An introductory section |$title| where on three lines the following
information is given:

\begin{enumerate}
\item A description of the run. This can be any text that fits on one line.
\item The name of the simulation. This name is used for all files that 
the simulation produces. These files differ from each other only by 
their extension.
\item The name of the basin. This is the basin file without the extension
|.bas|. The file must exist in the current directory.
\end{enumerate}

\item A section |$para| that contains all necessary parameters for the
simulation to be run. The only compulsory parameters are the ones that
specify the start of the simulation |itanf|, its end |itend|, its 
time step |idt| and a reference date (|date = yyyymmdd|). This is the
reference for all the time parameters used in the |str| file and in all 
the files provided as input to the simulation, as well as the output files.

\end{itemize}

All the time parameters can be specified in seconds from the reference
date or using a date label |'yyyy-mm-dd::HH:MM:SS'|. The parameters that specify
time steps can be prescribed both in seconds or using the following labels: 
|'Ns'|, |'Nm'|, |'Nh'|, |'Nd'|. 
Where |N| is a number, |s| means seconds, |m| minutes, |h| hours and |d| days.

In order to be more helpful, some more information must be added to the
|str| file. 
\Fig\ref{fig:example} shows an example of a typical input
parameter file, for 2D applications, and the use of the sections and definition of
parameters.

\begin{figure}[htbp]
\begin{alltt}
\input{example.str}
\end{alltt}
\caption{Example of a parameter input file ({\tt STR} file)}
\label{fig:example}
\end{figure}

As an example let's have a look on Fig. \figref{example}. 

% this from usage.tex

Here, two parameters that deal with the type of friction have been added.
|ireib| specifies the bottom friction formulation with its value equals to 
5 denoting a simple quadratic bulk formula (For the exact meaning of the
parameters, please refer to the appendix where all parameters
are listed.) The parameter |czdef| specifies the bottom drag coefficient value to be used.



%\begin{figure}
%%\begin{alltt}
%\includegraphics[width=12cm,height=18cm]{example3D.pdf}
%\includegraphics{example3D.pdf}
%%\end{alltt}
%\caption{Example of a 3D parameter input file ({\tt STR} file)}
%\label{fig:example3D}
%\end{figure}