
%------------------------------------------------------------------------
%
%    Copyright (C) 1985-2020  Georg Umgiesser
%
%    This file is part of SHYFEM.
%
%    SHYFEM is free software: you can redistribute it and/or modify
%    it under the terms of the GNU General Public License as published by
%    the Free Software Foundation, either version 3 of the License, or
%    (at your option) any later version.
%
%    SHYFEM is distributed in the hope that it will be useful,
%    but WITHOUT ANY WARRANTY; without even the implied warranty of
%    MERCHANTABILITY or FITNESS FOR A PARTICULAR PURPOSE. See the
%    GNU General Public License for more details.
%
%    You should have received a copy of the GNU General Public License
%    along with SHYFEM. Please see the file COPYING in the main directory.
%    If not, see <http://www.gnu.org/licenses/>.
%
%    Contributions to this file can be found below in the revision log.
%
%------------------------------------------------------------------------
\label{where} 

\subsection{From website}
From website shyfem.org you can download the latest stable version of the model.
The version released from the website is the stable one and is always updated to the latest release. 
\subsection{From GitHub}

%From |Google Drive| you can
%download the latest and older version of the model. In order to do so, please go to
%|https://drive.google.com/open?id=0B742mznAzyDPbGF2em5NMjZYdHc| and
%download the version you would like to install on your computer. Normally
%this will be the last available version.
Since SHYFEM is a model in constant development, for research purposes the user can download the develop versions, 
and also the latest stable one, from GitHub.


Please go to the |GitHub| website of the \shyfem{} model
|https://github.com/SHYFEM-model/shyfem| and navigate to |releases|.  
You either can visualize the releases or the tags (versions)
of the model and download them. Please see below for the difference
between tags and releases.

If you are a developer then you should install the |git|
versioning system that will give you direct access to the latest versions
of \shyfem{}.  Please see below how to do this.

Here is some information on how the various releases are managed and where to
download the latest version.

There are various types of versions:

\begin{itemize}

\item commit: commits are the smallest changes in the code base. 
Every time changes have been carried out, they will be committed to the repository. 
It is possible to see all the commits of the code
by typing |git-tags| which presents all the commits and also the tags,
which are explained here below.

\item version: the name version is just a shortcut for an existing tag 
or release that will have a version number associated.  
A commit has no version number and can therefore not
be identified in this way.

\item tag: tags are like commits, but a version number is given to
them. This means that these tags are more stable than simple commits. It
is always advisable to download tags in order to be able to easily refer
to the version number of \shyfem{}.

\item release: releases are nothing else than tags, but a name is also
given to this tag. This means that releases should be even more stable
than tags or commits. If you do not need a bleeding edge version, than
these are the versions that you should download.

\end{itemize}

For an easy use of github follow this procedure:
\begin{itemize}
\item to find commits and tags: use |gittags|. Its output provide the latest
 commits and tags (if applied). 

\item to see releases:  go to the github web page
(|https://github.com/SHYFEM-model/shyfem|). Click on |releases| and 
you can directly download the latest
version of \shyfem{}.

 \item to get access to the latest single commits, not available at |github| web page:
\subitem install |git| on your computer. 
\subitem go to the web page and click on |clone| to download the latest version with all
the versioning information included. 
\subitem Unpack the archive in any directory, and enter the newly created directory.
\subitem Type |git fetch| and |git pull| to have the newest commit of the code base.
\end{itemize}

 If you are a developer:
\begin{itemize}
\item you should have git installed on your computer.
\item If you want your changes in code being published do a |pull request| from the |github| website. 
\subitem Please be sure that you base your 
|pull request| on the latest commit (not tag or release). 
\subitem do a |git fetch| and |git pull|
\subitem check if your changes are compatible
\subitem do the |pull request|.
\end{itemize}

 All |pull requests| have to be based on the |develop| branch.

